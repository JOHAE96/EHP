
\documentclass[12pt,a4paper]{scrartcl}

\usepackage{fancyhdr}
\usepackage{times}
\usepackage[T1]{fontenc} 
\usepackage[latin1]{inputenc} 
\usepackage[ngerman]{babel}
\usepackage[right=3cm]{geometry}
\usepackage{listings} % for code
\lstset{language=C, numbers=left}

\parindent=0pt
\parskip=1em plus 2pt minus 1pt

\topskip-2cm
\textheight25cm
\pagestyle{fancy}
\fancyhead{}
\fancyfoot{}
\rhead{\thepage}

\begin{document}
\thispagestyle{empty}

\begin{center}
  \LARGE
  Praktische \"Ubungen - \\
  Experimentelle Hardwareprojekte \\
  \bigskip
  \Large 
  Versuchsprotokoll
\end{center}

\vspace{1em}
Versuch: S - SIMD 

Versuchsdatum und -zeit: 23. Mai 2018, 14 - 17 Uhr 

Betreuer: Daniel Walther

\vspace{1em}
Name, Studiengang, Mat.-Nr.: Alexander K\"uhnle, B.Sc. Informatik, 165692

Email: alexander.kuehnle@uni-jena.de

\vspace{1em}
Name, Studiengang, Mat.-Nr.: Mark Umnus, B.Sc. Informatik, 167419

Email: mark.umnus@uni-jena.de

\vspace*{1cm}
%% \mbox{}\\
\hrule
\vspace*{1cm}
{\Large  Eigenst\"andigkeitserkl\"arung }
 
Hiermit versichern wir, dass wir das Protokoll selbstst\"andig verfasst
und keine anderen Quellen und Hilfsmittel als die angegebenen benutzt 
haben. Im Falle einer Zuwiderhandlung erkennen wir an, dass unser Protokoll 
als nicht bestanden bewertet wird und damit das Modul ``Experimentelle 
Hardwareprojekte'' als nicht bestanden bewertet wird. \\
Dar\"uberhinaus ist uns klar, dass jede Zuwiderhandlung ausnahmslos dem 
Rechtsamt der FSU gemeldet wird, woraus weitere Konsequenzen resultieren 
k�nnen. \\

Unterschriften: \\ 
\hspace*{4cm} ........................................ 
\hspace{2cm} ........................................  \\

\hrule

\vspace*{0.3cm}
\textbf{Vom Betreuer auszuf�llen:}

Vorbereitung/Kolloquium:

Durchf�hrung:

Protokoll:

Gesamtbewertung:
\clearpage


% Hier geht das Protokoll los...

\section{Vorbereitung}
SIMD-Operationen werden heute \"uberall genutzt, wo viele Daten gleich verarbeitet werden m\"ussen, h\"aufig in Echtzeit.
Das gilt insbesondere in der Bild- und Videoverarbeitung, in der Kommunikation und im Gamingbereich.
Zur Vorbereitung auf das Praktikum haben wir uns mit dem Positivbeispiel AltiVec auseinandergesetzt und es mit der hier verwendeten SSE-Architektur verglichen.

Ziel des Praktikums war es, zu lernen die St\"arken von SIMD zu nutzen und die M�glichkeiten der compilerseitigen Optimierung kennenzulernen.

\section{Vorgehensweise}
Zum Erreichen dieses Ziels sollte der Bereich Videoverarbeitung als Beispiel dienen.
Dazu wurden zwei Videos in Form von Frames zu Verf\"ugung gestellt, die mit Fade-Ins und Fade-Outs modifiziert werden sollten.
Konkret war die vorgegebene Datei \texttt{simd.c} zu erweitern.
Als Hilfestellung waren in der \texttt{support.h}-Datei Funktionen definiert, die vom aufw\"andigen Laden und Speichern der Bilder abstrahiert haben.
Zun\"achst sollte ausgehend von einem schwarzen Bild ein Video A eingeblendet werden, anschlie\ss end in ein Video B \"ubergehen, welches schlie\ss lich in ein gr\"unes Bild ausgeblendet werden sollte.
Zum Abschluss jeder Teilaufgabe wurde das Makefile genutzt, um die Dateien zu kompilieren und den Film abzuspielen.
Final sollte ein Einblick in die Compilerausgabe gewonnen werden, indem der Assemblercode zu betrachten war.


\section{Erprobung}

\subsection{Anzeige eines schwarzen Bildes f\"ur zwei Sekunden}
In der ersten Aufgabe 3.1 musste lediglich der vorgegebene Code aktiviert werden, indem die Pr\"aprozessordirektive von 0 auf 1 gesetzt wurde.
Dieser Teil hatte den Effekt eine zweisek\"undige Sequenz mit einem schwarzen Bildschirm einzublenden.
Das Makefile konnte zun\"achst nicht ohne Weiteres genutzt werden, da der eingestellte Media Player \emph{mplayer} nicht installiert war.
Die entsprechende Zeile haben wir durch den Player \emph{totem} ge\"andert.

\subsection{Einblenden des Videos \emph{A} f\"ur zwei Sekunden}
Die zweite Aufgabe bestand darin, das Video A \"uber einen Zeitraum von 2 Sekunden einzublenden.
Dazu waren die Zeilen 14-16 von Listing \ref{a3.2} hinzuzuf\"ugen, der Rest war gegeben.

\begin{lstlisting}[caption=Fade-In von Video A \"uber 2 Sekunden, label=a3.2]
int frame;
float fade_duration = 2 * 25.0;

for (frame = 0; frame < (int) fade_duration; frame++)
{
   __m128 skalar;

    load_next_frame_from_videoA (inputframeA);
    skalar = _mm_set1_ps (((float) frame) / fade_duration);

    for (i=0; i < IMAGE_SIZE * 4; i+=4)
    {
        __m128 floatvektor, ergebnis;
        floatvektor = _mm_load_ps (&inputframeA[i]);
        ergebnis = _mm_mul_ps (skalar, floatvektor);
        _mm_store_ps (&outputframe[i], ergebnis);
    }

    save_frame (outputframe);
}
\end{lstlisting}

Die \"au\ss ere for-Schleife sorgt daf\"ur, dass insgesamt 50 Bilder f\"ur 2 Sekunden Video erzeugt werden.
In jeder Iteration wird das n\"achste Bild von Video A geladen und \texttt{skalar} auf den prozentualen Anteil der bereits vergangenen Frames gesetzt (Zeile 9).
Es folgt eine innere for-Schleife, die jedes Pixel durchl\"auft.
Die Laufvariable \texttt{i} nimmt dabei Vielfache von 4 an, da ein Pixel aus 4 float-Werten besteht, die sp\"ater gemeinsam geladen und verarbeitet werden.
Zeile 14 zeigt, wie das aktuelle Pixel des Bildes in der Variablen \texttt{floatvektor} gespeichert wird.
Diese wird in einem MMU-Register gehalten. % EDIT: UEberpruefen!
In Zeile 15 wird dann logisch eine Skalar-Vektor-Multiplikation ausgef\"uhrt.
Physisch werden dabei die Werte aus \texttt{floatvektor} jeweils mit einem eigenen Anteil multipliziert.
Das aktuelle Pixel wird in Zeile 16 zur\"uck in \texttt{outputframe} geschrieben, welches dann in Zeile 19 als Bilddatei auf die Festplatte geschrieben wird.

Nach dem Kompilieren wurde das gew\"unschte und erwartete Ergebnis erhalten.

\subsection{Anzeige des Videos \emph{A} f\"ur zwei Sekunden}
F\"ur diesen Teil konnte einfach der Beispielcode aus der Versuchsbeschreibung verwendet werden (Listing \ref{a3.3}).

\begin{lstlisting}[caption=Abspielen des Videos \emph{A}, label=a3.3]
fade_duration = 2.0 * 25.0;
for (frame = 0; frame < (int) fade_duration; frame++) {
    load_next_frame_from_videoA (inputframeA);
    save_frame(inputframeA);
}
\end{lstlisting}

\subsection{\"Uberblendung des Videos \emph{A} in Video \emph{B} f\"ur vier Sekunden}
In dieser Aufgabe war der Gro\ss teil schon vorgegeben, inklusive der Formel zur Berechnung.
Es mussten lediglich die Zeilen 6 und 7, sowie 15 bis 23 in Listing \ref{a3.4} erg\"anzt werden.

\begin{lstlisting}[caption=\"Uberblendung des Videos \emph{A} in Video {B}, label=a3.4]
fade_duration = 4.0 * 25.0;

for (frame = 0; frame < (int) fade_duration; frame++)
{
    __m128 alpha, beta;
    beta = _mm_set1_ps (((float) frame) / fade_duration);
    alpha = _mm_sub_ps (_mm_set1_ps (1.0f), beta);

    load_next_frame_from_videoA(inputframeA);
    load_next_frame_from_videoB(inputframeB);

    for (i=0; i < IMAGE_SIZE * 4; i+= 4)
    {
        __m128 vektorA, vektorB, ergebnis;
        vektorA = _mm_load_ps (&inputframeA[i]);
        vektorB = _mm_load_ps (&inputframeB[i]);

        __m128 betaB, alphaA;
        alphaA = _mm_mul_ps (alpha, vektorA);
        betaB = _mm_mul_ps (beta, vektorB);
        ergebnis = _mm_add_ps (betaB, alphaA);

        _mm_store_ps (&outputframe[i], ergebnis);
    }
    save_frame (outputframe);
}
\end{lstlisting}

An dieser Stelle sei darauf hingewiesen, dass wir abweichend von der Vorgabe $outputframe = \alpha \cdot inputframeA + \beta \cdot inputframeB$ mit $\alpha = (1 - \beta)$ berechnen, um $\beta$ von $0\rightarrow1$ laufen lassen zu k\"onnen und nicht umgekehrt.
W\"urde man die gegebene Formel mit $\beta$ von $0\rightarrow1$ nutzen, h\"atte man einen Sprung von Video \emph{A} nach Video \emph{B} mit anschlie\ss endem Fade-Over zur\"uck zu Video \emph{A}.

Die \"au\ss ere Schleife ist analog zu der in Listing \ref{a3.3}.
In Zeile 7 wird das vektorwertige \texttt{alpha} auf $1-\beta$ gesetzt.
Der Hauptteil geschieht in den Zeilen 19 bis 21, wo die Rechnung in zwei Schritten ausgef\"uhrt wird.
Zun\"achst werden \texttt{alphaA} (sprich: alpha mal A) und \texttt{betaB} nach der Fade-Formel von Aufgabe 3.2 berechnet.
Es folgt eine Summation der Teilergebnisse.
Der Rest (innere Schleife und das Laden und Schreiben der Input- und Outputdaten) erfolgt wie in Aufgabe 3.2.

\subsection{Anzeige des Videos \emph{B} f\"ur zwei Sekunden}
Dieser Teil ist analog zu Aufgabe 3.2:
\begin{lstlisting}[caption=Abspielen des Videos \emph{B}, label=a3.5]
fade_duration = 2.0 * 25.0;
for (frame = 0; frame < (int) fade_duration; frame++) {
    load_next_frame_from_videoA (inputframeB);
    save_frame(inputframeB);
}
\end{lstlisting}

\subsection{Ausblendung des Videos \emph{B} f\"ur zwei Sekunden nach Gr\"un}
Diese Aufgabe wurde im Prinzip mithilfe des Codes aus Listing \ref{a3.4} gel\"ost mit dem Unterschied, dass Video \emph{A} durch ein konstant gr\"unes Bild ersetzt wurde:

\begin{lstlisting}[caption=\"Uberblendung des Videos \emph{B} nach gr\"un, label=a3.6]
fade_duration = 2.0 * 25.0;

__m128 green;
green = _mm_set_ps (0.0f, 0.0f, 1.0f, 0.0f);
    
for (frame = 0; frame < (int) fade_duration; frame++)
{
    __m128 alpha, beta;
    beta = _mm_set1_ps (((float) frame) / fade_duration);
    alpha = _mm_sub_ps (_mm_set1_ps (1.0f), beta);

    load_next_frame_from_videoB(inputframeB);

    for (i=0; i < IMAGE_SIZE * 4; i+= 4)
    {
        __m128 vektorB, ergebnis;
        vektorB = _mm_load_ps (&inputframeB[i]);

        __m128 betagreen, alphaB;
        alphaB = _mm_mul_ps (alpha, vektorB);
        betagreen = _mm_mul_ps (beta, green);
        ergebnis = _mm_add_ps (betagreen, alphaB);
            
        _mm_store_ps (&outputframe[i], ergebnis);
    }
    save_frame(outputframe);
}
\end{lstlisting}

Die wichtigste Ver\"anderung steht damit in den Zeilen 3 und 4.
Die Farbe gr\"un wird per RGBA-Vektor dargestellt, dessen Komponenten jedoch in der Reihenfolge [A, B, G, R] eingegeben werden m\"ussen.
M\"ochte man die maximale Farbintensit\"at erreichen, muss man den entsprechenden float-Wert auf 1.0f setzen.
Man nutzt also eine relative Skala anstelle der Werte von 0 bis 255, da float-Werte auf dem Intervall $[0,\pm1]$ am genausten sind, w\"ahrend bei gr\"o\ss eren Zahlen die Abst\"ande zwischen den diskreten Werten des Datentyps float ebenfalls immer gr\"o\ss er werden.\\
Mit diesem gr\"unen Pixel wird nun die Fade-Over-Berechnung f\"ur Video \emph{B} durchgef\"uhrt, so als ob \texttt{green} ein komplettes Bild w\"are.

\section{Schlussfolgerungen}


\end{document}


