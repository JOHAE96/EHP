
\documentclass[12pt,a4paper]{scrartcl}

\usepackage{amsmath}
\usepackage{fancyhdr}
\usepackage{times}
\usepackage[T1]{fontenc}
\usepackage[latin1]{inputenc}
\usepackage[ngerman]{babel}
\usepackage[right=3cm]{geometry}
\usepackage{amsmath}
\usepackage{listings}
\lstdefinelanguage{PSpice}{
    keywords={.OPTIONS, .PROBE, .DC, .MODEL, .END},
    keywordstyle=\color{blue}\bfseries,
    comment=[l]{*}
    }
\lstset{language=PSpice,numbers=left}

\renewcommand\thesubsection{A\arabic{subsection}}

\parindent=0pt
\parskip=1em plus 2pt minus 1pt

\topskip-2cm
\textheight25cm
\pagestyle{fancy}
\fancyhead{}
\fancyfoot{}
\rhead{\thepage}

\begin{document}
\thispagestyle{empty}

\begin{center}
  \LARGE
  Praktische \"Ubungen - \\
  Experimentelle Hardwareprojekte \\
  \bigskip
  \Large
  Versuchsprotokoll
\end{center}

\vspace{1em}
Versuch: D - Digitalelektronik

Versuchsdatum und -zeit:14.06.2018, 10.00 Uhr bis 13.00 Uhr

Betreuer: Wolfgang Koch

\vspace{1em}
Name, Studiengang, Mat.-Nr.: Alexander K\"uhnle, B.Sc. Informatik, 165692

Email: alexander.kuehnle@uni-jena.de

\vspace{1em}
Name, Studiengang, Mat.-Nr.: Mark Umnus, B.Sc. Informatik, 167419

Email: mark.umnus@uni-jena.de

\vspace*{1cm}
%% \mbox{}\\
\hrule
\vspace*{1cm}
{\Large  Eigenst\"andigkeitserkl\"arung }

Hiermit versichern wir, dass wir das Protokoll selbstst\"andig verfasst
und keine anderen Quellen und Hilfsmittel als die angegebenen benutzt
haben. Im Falle einer Zuwiderhandlung erkennen wir an, dass unser Protokoll
als nicht bestanden bewertet wird und damit das Modul ``Experimentelle
Hardwareprojekte'' als nicht bestanden bewertet wird. \\
Dar\"uberhinaus ist uns klar, dass jede Zuwiderhandlung ausnahmslos dem
Rechtsamt der FSU gemeldet wird, woraus weitere Konsequenzen resultieren
k\"onnen. \\

Unterschriften: \\
\hspace*{4cm} ........................................
\hspace{2cm} ........................................  \\

\hrule

\vspace*{0.3cm}
\textbf{Vom Betreuer auszuf\"ullen:}

Vorbereitung/Kolloquium:

Durchf\"uhrung:

Protokoll:

Gesamtbewertung:
\clearpage


% Hier geht das Protokoll los...

\section{Vorbereitung}
In diesem Versuch soll das Verhalten von MOS-Feldeffekttransitoren untersucht werden.
Daf\"ur werden die Transitoren am Computer simuliert und getestet, um anschlie\ss end auf einem Steckbrett in Hardware realisiert zu werden.
In der Vorbereitung auf den Versuch haben wir uns noch einmal mit elektrotechnischen Grundlagen auseinandergesetzt.
Daneben haben wir Vor\"uberlegungen angestellt und unsere Versuchsparameter ausgerechnet, die von unserer Gruppe (Nummer 2) und vom Datum abhingen.
Insgesamt sind f\"ur diesen Versuch folgende Werte relevant:

\begin{align} \label{eq:geg}
&U_V = 5V \nonumber \\
&U_{th} = 1.4V \nonumber \\
&K = 10^{-3} \cdot (4.5 + Tag/10) \frac{A}{V^2} = 10^{-3} \cdot (4.5 + 14/10) \frac{A}{V^2} = 5.9 \cdot 10^{-3} \frac{A}{V^2} \nonumber \\
&R = 400\Omega + Gruppe \cdot 100\Omega = 400\Omega + 200\Omega = 600 \Omega
\end{align}

Anschlie\ss end haben wir die \"Ubertragungskennlinie f\"ur diese Schaltung in mehreren Schritten berechnet.
Wir haben folgende Formeln verwendet:

\begin{equation} \label{eq:UDS}
    U_{DS} = U_V - I_D \cdot R
\end{equation}

\begin{equation} \label{eq:Up}
    U_p = U_{GS} - U_{th}
\end{equation}

\begin{align} \label{eq:ID}
&I_D = \left\{
  \begin{array}{lr}
    0  & U_{GS} < U_{th}\\
    K((U_{GS} - U_{th})U_{DS} - \frac{1}{2}U_{GS}^2) & U_{DS} < U_p\\
    \frac{1}{2}K(U_{GS} - U_{th})^2 & U_{GS} \geq U_{th} \wedge U_{DS} \geq U_p
  \end{array}
\right.
\end{align}

\begin{align} \label{eq:pq}
    & x^2 -px +q = 0 \nonumber \\
    \Leftrightarrow & x_{1,2} = -\frac{p}{2} \pm \sqrt{\frac{p^2}{4} - q}
\end{align}

F\"ur die \"Ubertragungskennlinie bedeutet dies, dass $I_D = 0$ gilt, bis $U_{GS} = U_{th}$, da davor der Widerstand des Transistors gegen unendlich geht.
Sobald $U_{GS}$ \"uber $U_{th}$ steigt, flie\ss t ein Strom und es tritt eine Abschn\"urung ein.
Der Grund daf\"ur ist, dass durch die Versorgungsspannung ein Potenzial am Drain anliegt, das $U_{GS}$ entgegenwirkt und den n-Kanal abschn\"urt.
Dieser Effekt nimmt mit steigendem Strom zu.
Im Folgenden soll daher der Abschn\"urpunkt $U_{ab} = U_{GS}$ berechnet werden, so dass $U_{DS} = U_p$.
\begin{align*}
&U_V - I_D \cdot R \overset{(\ref{eq:UDS})}{=} \boldsymbol{U_{DS}} \overset{!}{=} U_p \overset{(\ref{eq:Up})}{=} U_{GS} - U_{th} = \underbrace{U_{ab} - U_{th}}_{=:x}\\
&\overset{(\ref{eq:ID})}{\Rightarrow} U_V - \frac{1}{2}Kx^2 \cdot R = x\\
&\Leftrightarrow -\frac{1}{2}Kx^2 \cdot R -x +U_V = 0 \\
&\Leftrightarrow x^2 + \frac{2}{KR}x -\frac{2U_V}{KR} = 0 \\
&\underset{x \geq 0}{\Rightarrow} x \overset{(\ref{eq:pq})}{=} -\frac{1}{KR} + \sqrt{\frac{1}{(KR)^2} + \frac{2U_V}{KR}} \approx 1.42182 V \\
&\underline{U_{ab} = x + U_{th} \approx 2.82182 V}
\end{align*}

In der Berechnung wurde der dritte Fall aus (\ref{eq:ID}) verwendet, da per Annahme $U_{DS} = U_p$ galt und $U_{GS}$ sinnvollerweise gr\"o\ss er als $U_{th}$ sein sollte.

Um die \"Ubertragungskennlinie tats\"achlich zu zeichnen, haben wir in Tabelle \ref{tab:wertevorb} sechs Werte von $I_D = f(U_{GS})$ und $U_a = U_{DS} = f(U_{GS})$ f\"ur $U_{th} \leq U_{GS} \leq U_{ab}$ berechnet.
An dieser Stelle nutzt man wie eben auch Fall 3 aus (\ref{eq:ID}), da $U_{DS} \geq U_{GS} - U_{th}$ gilt.
$U_{DS}$ berechnet man mit Formel (\ref{eq:UDS}).

\begin{table}[h]
    \centering
    \begin{tabular}{l|l|l}
    \hline
    $U_{GS}$ in V     & $I_D$ in mA  & $U_{DS}$ in V \\
    \hline
    1.5               & 0.0295       & 4.9823        \\
    1.75              & 0.3614       & 4.7832        \\
    2.0               & 1.062        & 4.3628        \\
    2.25              & 2.131        & 3.7214        \\
    2.5               & 3.5695       & 2.8583        \\
    $U_{ab}$          & 5.9636       & 1.4218        \\
    \hline
    \end{tabular}
    \caption{$I_D = f(U_{GS})$ und $U_{DS} = f(U_{GS})$ f\"ur $U_{GS} < U_{ab}$}
    \label{tab:wertevorb}
\end{table}

Als Best\"atigung ist $U_{DS} = U_{GS} - U_{th}$ bei $U_{GS} = U_{ab}$.
Wird $U_{GS} > U_{ab}$, dann ist $U_{DS} < U_{GS} - U_{th}$ und man kann Fall 2 aus Formel (\ref{eq:ID}) verwenden.
Es gilt:

\begin{align}
U_{DS} & \overset{(\ref{eq:UDS})}{=} U_V - I_D \cdot R \overset{(\ref{eq:ID})}{=} U_V - KR((U_{GS} - U_{th})U_{DS} - \frac{1}{2}U_{DS}^2) \nonumber \\
\Leftrightarrow 0 & = U_V - KR(U_{GS} - U_{th} - 1)U_{DS} - \frac{1}{2}U_{DS}^2 \nonumber \\
\Leftrightarrow 0 & = U_{DS}^2 + 2(U_{GS} - U_{th} - \frac{1}{KR})U_{DS} - \frac{2U_V}{KR} \nonumber \\
\underset{U_{DS} \geq 0}{\Rightarrow} U_{DS} & \overset{(\ref{eq:pq})}{=} (-U_{GS} + U_{th} + \frac{1}{KR}) + \sqrt{(U_{GS} - U_{th} - \frac{1}{KR})^2 + \frac{2U_V}{KR}}
\end{align}

Setzt man dieses Ergebnis nun in (\ref{eq:UDS}) ein und stellt es nach $I_D$ um, so kann man die Wertetabelle f\"ur $U_{GS} \geq U_{ab}$ vervollst\"andigen.
Wir beschr\"anken uns hier zur sp\"ateren Kontrolle auf $U_{GS} = 5V$.
Es ergibt sich $U_{DS} \approx 0.40145868$V $\Rightarrow I_D \approx 7.664$mA.

Nun bleibt es, den Restwiderstand und die Verlustleistung des vollst\"andig ge\"offneten Transistors zu bestimmen.
Da es sich bei diesem Aufbau um eine Reihenschaltung handelt, ist der Gesamtwiderstand die Summe der Einzelwiderst\"ande.
Der Gesamtwiderstand der Schaltung l\"asst sich einfach mit $R = \frac{U}{I}$ bestimmen.
Zusammen ergibt sich:

\begin{align*}
R_{Transistor} & = R_{ges} - R \\
& = \frac{5\text{V}}{0.007664\text{A}} - 600\Omega\\
& \underline{\approx 52.4\Omega}
\end{align*}

Zudem bel\"auft sich die Verlustleistung auf 
\begin{align*}
P_{Transistor} & = I_D \cdot U_{DS} \\
& \approx 0.007664\text{A} \cdot 0.40145868\text{V} \\
& \underline{\approx 3.07678\text{mW}}
\end{align*}

\section{Vorgehensweise}
Das Verhalten der Hardware wurde mit Hilfe eines DA/AD-Wandlers gemessen.
Die Simulation wurde mit Hilfe von PSpice durchgef\"uhrt.

\section{Erprobung}
\subsection{Ausgangskennlinie simulieren}
In der ersten Aufgabe sollte die Ausgangskennlinie $I_D = f(U_{DS}$ simuliert werden.
Dies sollte zun\"achst f\"ur $U_{GS} = 3$V und anschlie\ss end f\"ur $U_{GS} \in \{0\text{V}, ..., 5\text{V}\}$ geschehen.
Abbildung 2, die die Ergebnisse zeigt, enth\"alt der Einfachheit halber gleich alle Ausgangskennlinien.
Der zugeh\"orige PSpice-Quellcode befindet sich im Anhang in Listing \ref{lst:a1}.
Es wurden jeweils die Abschn\"urpunkte gekennzeichnet, ab denen der Strom nicht mehr zunimmt.
Diese lassen sich auch durch $U_{GS}-U_{th}$ berechnen.
Wie man sieht, ist der maximal m\"ogliche Strom jeweils abh\"angig von der Steuerspannung $U_{GS}$.
Das liegt daran, dass das elektrische Feld, das am Gate-Anschluss erzeugt wird, dann entsprechend st\"arker ist.

Die Linien f\"ur $U_{GS} = 0\text{V}, 1$V sind auf der Abbildung nicht zu sehen, sie liegen auf der x-Achse.
F\"ur diese gilt n\"amlich $U_{GS} < U_{th}$, es flie\ss t also unabh\"angig von der Drain-Source-Spannung kein Strom.

Die Str\"ome haben im Abschn\"urbereich quadratische Abst\"ande, wie man sich leicht klarmachen kann, indem man die Abst\"ande der Abst\"ande berechnet, die dann konstant sind:
$$diff(diff(38.232, 19.942, 7.552, 1.062)) = diff(18.29, 12.39, 6.49) = (5.9, 5.9)$$
Wie man sieht, ist dieser konstante Faktor gerade das Tausendfache der zuvor berechneten Steilheit K.
Da Transistorenhersteller diese Steilheit beeinflussen k\"onnen, haben sie auch in der Hand, um wie viel h\"oher der Abschn\"urpunkt bei h\"oherer Spannung liegt.
Diese Steilheit kann man auch so betrachten:
H\"alt man $U_{DS}$ konstant und erh\"oht stattdessen $U_{GS}$, so steigt der flie\ss ende Strom quadratisch.
Diesen Effekt werden wir in A2 detaillierter untersuchen.

\subsection{\"Ubertragungskennlinie simulieren}
In diesem Abschnitt wurden in PSpice sowohl die \"Ubertragungskennlinie $I_D=f(U_{GS})$ (Abbildung 3) als auch die $U_{DS}=f(U_{GS})$ (Abbildung 4) simuliert.
Auf Abbildung 3 erkennt man gut, wie kein Strom flie\ss t, wenn $U_{GS} < 1.4=U_{th}$.
Zudem ist zu sehen, wie die Steigung des Graphen stark abf\"allt, wenn $U_{GS} \geq 2.823$, was dem zuvor berechneten Abschn\"urpunkt entspricht.
Insgesamt stimmt diese Kurve gut mit der in der Vorbereitung berechneten Kennlinie \"uberein.
Auf Abbildung 3 wurde der Abschn\"urpunkt durch den Schnittpunkt mit der Widerstandsgeraden markiert, die durch $|U_{GS} - U_{th}|$ gegeben ist.
Aus den Abbildungen kann man ablesen, dass f\"ur $U_{GS} = 5$V die Werte $I_D = 7.6896$mA und $U_{DS} = 0.38265$V gelten.
Daraus berechnet sich der Restwiderstand wie folgt:
\begin{align*}
R_{Transistor} & = \frac{U_V}{I_D} -R\\
& = \frac{5\text{V}}{0.0076896\text{A}} - 600\Omega \\
& \underline{ \approx 50.2289\Omega}
\end{align*}

Wie die beiden Diagramme zusammenh\"angen, wurde in der vorherigen Aufgabe bereits kurz angerissen.
Beide stellen den Zusammenhang der Werte $U_{GS}$, $U_{DS}$ und $I_D$ dar.
Deshalb enthalten sie \"ahnliche Informationen.
Die quadratischen Abst\"ande der \"Aste in Abbildung 2 entsprechen dem quadratischen Anstieg in Abbildung 3, was wiederum der Steilheit K entspricht.

\subsection{\"Ubertragungskennlinie messen}
In dieser Aufgabe wurde erstmals das Experimentierboard genutzt, um die vorher simulierten \"Ubertragungskennlinien zu messen.
Verwendet wurden dabei ein 10k$\Omega$-Widerstand und ein n-Schicht-MOSFET.
Mit rot wurde in den Abbildungen 5 und 6 jeweils die theoretische Kurve gezeichnet und die gemessene mit blau.
Man sieht, dass die Messwerte erstaunlich gut mit der theoretischen Kurve \"ubereinstimmen.

\subsection*{Vertiefung: Arbeitsgerade}
In dieser Vertiefungsaufgabe sollte die Arbeitsgerade mit einem Widerstand von 100$\Omega$ in Abbildung 2 gezeichnet werden.
Wir haben mit einem Widerstand von 200$\Omega$ simuliert, um \"ubersichtlichere Ergebnisse zu erhalten.
Die Gleichung der Arbeitsgerade l\"asst sich durch den Vorwiderstand und der an ihm abfallenden Spannung berechnen:

\begin{align*}
I_{AG} & = \frac{U_V - U_{DS}}{R} \\ 
& = \frac{5\text{V} - U_{DS}}{200\Omega}
\end{align*}

Die Arbeitspunkte sind in Tabelle \ref{tab:arbeitspunkte} zusammen mit der Leistung an diesen aufgef\"uhrt.

\begin{table}[h]
    \centering
    \begin{tabular}{l|l|l|l}
    \hline
    $U_{GS}$ in V & $U_{DS}$ in V     & $I_AG$ in mA  & P in mW \\
    \hline
    5             & 1.0871            & 19.604        & 21.3115 \\
    4             & 1.6               & 16.992        & 27.1872 \\
    3             & 3.4936            & 7.552         & 26.2837 \\
    2             & 4.79              & 1.0620        & 5.08698 \\
    \hline
    \end{tabular}
    \caption{Arbeitspunkte der Schaltung aus Vorwiderstand und n-MOSFET mit Verlustleistung}
    \label{tab:arbeitspunkte}
\end{table}

Diese Vertiefungsaufgabe sollte zeigen, dass bei einem n-MOSFET mit Vorwiderstand fast permanent \"uber 20mW geleistet werden.
Dadurch entsteht Abw\"arme, die vermieden werden kann und sollte.
Eine L\"osung hei\ss t CMOS und wird im Folgenden untersucht.

(Die Probe mussten wir aus Zeitgr\"unden auslassen.)

\subsection{\"Ubertragungskennlinie CMOS-Inverter messen}
\label{invertermessen}
Die Idee hinter CMOS ist es, den Vorwiderstand durch einen komplement\"aren MOSFET zu ersetzen.
Dadurch kann verhindert werden, dass zur Betriebszeit Strom flie\ss t.
Die einzigen Momente mit Stromfluss und damit Verlustleistung sind dann die Umschaltmomente.
Nun haben wir auf dem Experimentierboard einen CMOS-Inverter aufgebaut, um hier die \"Ubertragungskennlinie $U_a=f(U_{GS})$ zu messen.
In der zugeh\"origen Abbildung 7 sollten dann die Logikpegel eingetragen werden, also ab wann die Spannung als High bzw. Low erkannt wird.
Wir haben 1.6V und 3.2V gew\"ahlt, da die Spannungen dar\"uber (bzw. darunter f\"ur High) keine deutlichen \"Anderungen mehr zu beobachten sind.
Spannungen dazwischen gelten als undefiniert.
Im Vergleich mit Abbildung 4 f\"allt der deutlich steilere Spannungsabfall von $U_{DS}$ auf, der durch den komplement\"aren Transistor bedingt ist.
Dieser hat n\"amlich f\"ur Spannungen $U_{GS}$, die gr\"o\ss er als die gesch\"atzten 3.2V sind, einen Widerstand, der gegen unendlich geht.

\subsection{Strom im CMOS-Inverter simulieren}
In dieser Aufgabe haben wir den CMOS-Inverter in PSpice beschrieben und den Strom $I_D=f(U_{GS})$ simuliert.
Das Ergebnis ist auf Abbildung 8 zu sehen, der Quellcode im Anhang in Listing \ref{lst:inverter}.
Man sieht, dass Strom nur flie\ss t, wenn $1.4V < U_{GS} < 1.9V$.
Letzteres ist dabei 5V-3.1V, was die Threshold-Spannung des p-MOSFET ist.
Zus\"atzlich kann man erkennen, dass selbst die maximale Stromst\"arke nur im $\mu$A-Bereich liegt.
Auff\"allig ist, dass der Peak nicht genau in der Mitte zwischen 1.4V und 1.9V liegt, sondern weiter auf der linken Seite.
Der Grund daf\"ur ist, dass die Steilheit des n-MOSFET gr\"o\ss er ist als die des p-MOSFET.

\subsection{\emph{NAND-Gatter} oder NOR-Gatter aufbauen}
In dieser Aufgabe haben wir ein NAND-Gatter auf dem Steckbrett realisiert.
Ein NAND-Gatter mit zwei Eing\"angen gibt genau dann den logischen Wert FALSCH aus, wenn beide Eing\"ange WAHR sind.
Unsere Steckverbindungen sind auf Abbildung xy+1 zu sehen.
Die \texttt{AO}-Anschl\"usse entsprechen dabei den Inputs, der Output wird an \texttt{AI 0} gemessen.
Wir haben zwei Messreihen erstellt, einmal mit unseren in \ref{invertermessen} bestimmten Logikpegeln und einmal mit 5V/0V als Abgleich.
Unsere Ergebnisse sind in Tabelle \ref{tab:nand} dargestellt.

\section{Schlussfolgerungen}
In diesem Versuch wurde gezeigt, dass...
Wir haben gelernt, wie man die W\"armeverlust durch den Vorwiderstand mit Hilfe von CMOS vermeiden kann.
Das ist die Grundlage f\"ur die moderne Digitalelektronik, da anderenfalls zu viel Abw\"arme entstehen w\"urde.
Des Weiteren haben wir gesehen, dass es bei CMOS beim Umschalten kurzzeitig einen Strom gibt, der aber angesichts der Milliarden von Transistoren stets beachtet werden muss.
Insofern hat uns dieser Versuch wichtige Einblicke erm\"oglicht und wir haben eine Menge Neues erfahren.

=> Schwierigkeiten

\section{Anhang}
\lstinputlisting[caption=PSpice-Quelltext f\"ur A1, label=lst:a1]{a1.cir}
\lstinputlisting[caption=PSpice-Quelltext f\"ur A5, label=lst:inverter]{a5.cir}

\end{document}
