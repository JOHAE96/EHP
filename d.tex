
\documentclass[12pt,a4paper]{scrartcl}

\usepackage{amsmath}
\usepackage{fancyhdr}
\usepackage{times}
\usepackage[T1]{fontenc}
\usepackage[latin1]{inputenc}
\usepackage[ngerman]{babel}
\usepackage[right=3cm]{geometry}
\usepackage{amsmath}

\parindent=0pt
\parskip=1em plus 2pt minus 1pt

\topskip-2cm
\textheight25cm
\pagestyle{fancy}
\fancyhead{}
\fancyfoot{}
\rhead{\thepage}

\begin{document}
\thispagestyle{empty}

\begin{center}
  \LARGE
  Praktische \"Ubungen - \\
  Experimentelle Hardwareprojekte \\
  \bigskip
  \Large
  Versuchsprotokoll
\end{center}

\vspace{1em}
Versuch: D - Digitalelektronik

Versuchsdatum und -zeit:14.06.2018, 10.00 Uhr bis 13.00 Uhr

Betreuer: Wolfgang Koch

\vspace{1em}
Name, Studiengang, Mat.-Nr.: Alexander K\"uhnle, B.Sc. Informatik, 165692

Email: alexander.kuehnle@uni-jena.de

\vspace{1em}
Name, Studiengang, Mat.-Nr.: Mark Umnus, B.Sc. Informatik, 167419

Email: mark.umnus@uni-jena.de

\vspace*{1cm}
%% \mbox{}\\
\hrule
\vspace*{1cm}
{\Large  Eigenst\"andigkeitserkl\"arung }

Hiermit versichern wir, dass wir das Protokoll selbstst\"andig verfasst
und keine anderen Quellen und Hilfsmittel als die angegebenen benutzt
haben. Im Falle einer Zuwiderhandlung erkennen wir an, dass unser Protokoll
als nicht bestanden bewertet wird und damit das Modul ``Experimentelle
Hardwareprojekte'' als nicht bestanden bewertet wird. \\
Dar\"uberhinaus ist uns klar, dass jede Zuwiderhandlung ausnahmslos dem
Rechtsamt der FSU gemeldet wird, woraus weitere Konsequenzen resultieren
k\"onnen. \\

Unterschriften: \\
\hspace*{4cm} ........................................
\hspace{2cm} ........................................  \\

\hrule

\vspace*{0.3cm}
\textbf{Vom Betreuer auszuf\"ullen:}

Vorbereitung/Kolloquium:

Durchf\"uhrung:

Protokoll:

Gesamtbewertung:
\clearpage


% Hier geht das Protokoll los...

\section{Vorbereitung}
In diesem Versuch sollte das Verhalten von MOS-Feldeffekttransitoren untersucht werden.
Daf\"ur wurden die Transitoren am Computer simuliert und getestet, um anschlie\ss end auf einem Steckbrett in Hardware realisiert zu werden.
In der Vorbereitung für den Versuch haben wir uns noch einmal mit elektrotechnischen Grundlagen auseinandergesetzt.
Daneben haben wir Vor\"uberlegungen angestellt und unsere Versuchsparameter ausgerechnet, die von unserer Gruppe (Nummer 2) und vom Datum abhingen.
Insgesamt sind f\"ur diesen Versuch folgende Werte relevant:
\begin{align} \label{eq:geg}
&U_V = 5V \nonumber \\
&U_{th} = 1.4V \nonumber \\
&K = 10^{-3}*(4.5 + Tag/10) \frac{A}{V^2} = 10^{-3}*(4.5 + 14/10) \frac{A}{V^2} = 5.9*10^{-3} \frac{A}{V^2} \nonumber \\
&R = 400\Omega + Gruppe*100\Omega = 400\Omega + 200\Omega = 600 \Omega
\end{align}
Anschlie\ss end haben wir die \"Ubertragungskennlinie f\"ur diese Schaltung in mehreren Schritten berechnet.
Wir haben folgende Formeln verwendet:
\begin{equation} \label{eq:UDS}
U_{DS} = U_V - I_D * R
\end{equation}
\begin{equation} \label{eq:Up}
U_p = U_{GS} - U_{th}
\end{equation}
\begin{align} \label{eq:ID}
&I_D = \left\{
  \begin{array}{lr}
    0  & U_{GS} < U_{th}\\
    K((U_{GS} - U_{th})U_{DS} - \frac{1}{2}U_{GS}^2 & U_{DS} < U_p\\
    \frac{1}{2}K(U_{GS} - U_{th})^2 & U_{GS} \geq U_{th} \wedge U_{DS} \geq U_p
  \end{array}
\right.
\end{align}

F\"ur die \"Ubertragungskennlinie bedeutet dies, dass $I_D = 0$ gilt, bis $U_{GS} = U_{th}$, da davor der Widerstand des Transistors gegen unendlich geht.
Sobald $U_{GS}$ \"uber $U_{th}$ steigt, flie\ss t ein Strom und es tritt eine Abschn\"urung ein.
Der Grund daf\"ur ist, dass durch die Versorgungsspannung ein Potenzial am Drain anliegt, das $U_{GS}$ entgegenwirkt und den n-Kanal abschn\"urt.
Dieser Effekt nimmt mit steigendem Strom zu.
Im Folgenden soll daher der Abschn\"urpunkt $U_{ab} = U_{GS}$ berechnet werden, so dass $U_{DS} = U_p$.
\begin{align*}
&U_V - I_D*R \overset{(\ref{eq:UDS})}{=} \boldsymbol{U_{DS}} \overset{!}{=} U_p \overset{(\ref{eq:Up})}{=} U_{GS} - U_{th} = \underbrace{U_{ab} - U_{th}}_{=:x}\\
&\overset{(\ref{eq:ID})}{\Rightarrow} U_V - \frac{1}{2}Kx^2*R = x\\
&\Leftrightarrow -\frac{1}{2}Kx^2*R -x +U_V = 0 \\
&\Leftrightarrow x^2 + \frac{2}{KR}x -\frac{2U_V}{KR} = 0 \\
&\underset{x \geq 0}{\Rightarrow} x \overset{p-q-Formel}{=} -\frac{1}{KR} + \sqrt{\frac{1}{(KR)^2} + \frac{2U_V}{KR}} \approx 1.42182 V \\
&\underline{U_{ab} = x + U_{th} \approx 2.82182 V}
\end{align*}

In der Berechnung wurde der dritte Fall aus (\ref{eq:ID}) verwendet, da per Annahme $U_{DS} = U_p$ galt und $U_{GS}$ sinnvollerweise gr\"o\ss er als $U_{th}$ sein sollte.


\section{Vorgehensweise}
Das Verhalten der Hardware wurde mit Hilfe eines DA/AD-Wandlers gemessen.
Die Simulation wurde mit Hilfe von PSpice durchgef\"uhrt.

\section{Erprobung}
\subsection{Ausgangskennlinie simulieren}
Die Ergebnisse die Simulation der Ausgangskennlinien sind auf Abbildung 1 zu sehen.
Es wurden jeweils die Abschn\"urpunkte gekennzeichnet, ab denen der Strom nicht mehr zunimmt.
Diese lassen sich auch durch $U_{GS}-U_{th}$ berechnen.
Wie man sieht, ist der maximal m\"ogliche Strom jeweils abh\"angig von der Steuerspannung $U_{GS}$.
Das liegt daran, dass das elektrische Feld, das am Gate-Anschluss erzeugt wird, dann entsprechend st\"arker ist.

Die Str\"ome haben im Abschn\"urbereich quadratische Abst\"ande, wie man sich leicht klarmachen kann, indem man die Abst\"ande der Abst\"ande berechnet, die dann konstant sind:
$$diff(diff(38.232, 19.942, 7.552, 1.062)) = diff(18.29, 12.39, 6.49) = (5.9, 5.9)$$
Wie man sieht, ist dieser konstante Faktor gerade das Tausendfache der zuvor berechneten Steilheit K.
Da Transistorenhersteller diese Steilheit beeinflussen k\"onnen, haben sie auch in der Hand, um wie viel h\"oher der Abschn\"urpunkt bei h\"ohrer Spannung liegt.

\subsection{\"Ubertragungskennlinie simulieren}
In diesem Abschnitt wurden in PSpice sowohl die \"Ubertragungskennlinie $I_D=f(U_{GS})$ (Abbildung 2) als auch die $U_{DS}=f(U_{GS})$ (Abbildung 3) simuliert.
Auf Abbildung 2 erkennt man gut, wie kein Strom flie\ss t, wenn $U_{GS} < 1.4=U_{th}$.
Zudem ist zu sehen, wie die Steigung des Graphen stark abf\"allt, wenn $U_{GS} \geq 2.823$, was dem zuvor berechneten Abschn\"urpunkt entspricht.

Auf Abbildung 3 wurde der Abschn\"urpunkt durch den Schnittpunkt mit der Widerstandsgeraden markiert.

\subsection{\"Ubertragungskennlinie messen}
In dieser Aufgabe wurde erstmals das Experimentierboard genutzt, um die vorher simulierten \"Ubertragungskennlinien zu messen.
Mit rot wurde jeweils die theoretische Kurve gezeichnet und die gemessene mit blau.
Man sieht, dass die Messwerte erstaunlich gut mit der theoretischen Kurve \"ubereinstimmt.

\subsection{\"Ubertragungskennlinie CMOS-Inverter messen}
Nun haben wir auf dem Experimentierboard einen CMOS-Inverter aufgebaut, um hier die \"Ubertragungskennlinie $U_a=f(U_{GS})$ zu messen.
In der zugeh\"origen Abbildung 6 sollten dann die Logikpegel eingetragen werden, also ab wann die Spannung als High bzw. Low erkannt wird.
Wir haben 1.6V und 3.2V gew\"ahlt, da die Spannungen dar\"uber (bzw. darunter f\"ur High) keine deutlichen \"Anderungen mehr zu beobachten sind.
Spannungen dazwischen werden als verboten definiert.

\subsection{Strom im CMOS-Inverter simulieren}
In dieser Aufgabe haben wir den CMOS-Inverter auch in PSpice beschrieben und den Strom $I_D=f(U_{GS})$ simuliert.
Das Ergebnis ist auf Abbildung 7 zu sehen.
Man sieht, dass Strom nur flie\ss t, wenn $1.4V < U_{GS} < 1.9V$.
Letzteres ist dabei 5V-3.1V, was die Threshold-Spannung des PMOSFET ist.
Auff\"allig ist, dass der Peak nicht genau in der Mitte zwischen 1.4V und 1.9V liegt, sondern weiter auf der linken Seite.
Der Grund daf\"ur ist, dass die Steilheit des NMOSFET gr\"o\ss er ist als die des PMOSFET.

\section{Schlussfolgerungen}
In diesem Versuch wurde gezeigt, dass...
Wir haben gelernt, wie man die W\"armeverlust durch den Vorwiderstand mit Hilfe von CMOS vermeiden kann.
Das ist die Grundlage f\"ur die moderne Digitalelektronik, da anderenfalls zu viel Abw\"arme entstehen w\"urde.
Des Weiteren haben wir gesehen, dass es bei CMOS beim Umschalten kurzzeitig einen Strom gibt, der aber angesichts der Milliarden von Transistoren stets beachtet werden muss.
Insofern hat uns dieser Versuch wichtige Einblicke erm\"oglicht und wir haben eine Menge Neues erfahren.

\end{document}
