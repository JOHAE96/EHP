
\documentclass[12pt,a4paper]{scrartcl}

\usepackage{amsmath}
\usepackage{fancyhdr}
\usepackage{times}
\usepackage[T1]{fontenc} 
\usepackage[latin1]{inputenc} 
\usepackage[ngerman]{babel}
\usepackage[right=3cm]{geometry}
\usepackage{listings} % for code
\lstset{language=C, numbers=left}

\parindent=0pt
\parskip=1em plus 2pt minus 1pt

\topskip-2cm
\textheight25cm
\pagestyle{fancy}
\fancyhead{}
\fancyfoot{}
\rhead{\thepage}

\begin{document}
\thispagestyle{empty}

\begin{center}
  \LARGE
  Praktische \"Ubungen - \\
  Experimentelle Hardwareprojekte \\
  \bigskip
  \Large 
  Versuchsprotokoll
\end{center}

\vspace{1em}
Versuch: G1 - Parallelrechner - Grafikkarten - Implementierung

Versuchsdatum und -zeit: 30. Mai 2018, 14 - 17 Uhr

Betreuer: Ralf Seidler

\vspace{1em}
Name, Studiengang, Mat.-Nr.: Alexander K\"uhnle, B.Sc. Informatik, 165692

Email: alexander.kuehnle@uni-jena.de

\vspace{1em}
Name, Studiengang, Mat.-Nr.: Mark Umnus, B.Sc. Informatik, 167419

Email: mark.umnus@uni-jena.de

\vspace*{1cm}
%% \mbox{}\\
\hrule
\vspace*{1cm}
{\Large  Eigenst\"andigkeitserkl\"arung }
 
Hiermit versichern wir, dass wir das Protokoll selbstst\"andig verfasst
und keine anderen Quellen und Hilfsmittel als die angegebenen benutzt 
haben. Im Falle einer Zuwiderhandlung erkennen wir an, dass unser Protokoll 
als nicht bestanden bewertet wird und damit das Modul ``Experimentelle 
Hardwareprojekte'' als nicht bestanden bewertet wird. \\
Dar\"uberhinaus ist uns klar, dass jede Zuwiderhandlung ausnahmslos dem 
Rechtsamt der FSU gemeldet wird, woraus weitere Konsequenzen resultieren 
k\"onnen. \\

Unterschriften: \\ 
\hspace*{4cm} ........................................ 
\hspace{2cm} ........................................  \\

\hrule

\vspace*{0.3cm}
\textbf{Vom Betreuer auszuf\"ullen:}

Vorbereitung/Kolloquium:

Durchf\"uhrung:

Protokoll:

Gesamtbewertung:
\clearpage


% Hier geht das Protokoll los...

\section{Vorbereitung}


\section{Vorgehensweise}
Dazu soll in diesem Versuch eine Grafikkarte so programmiert werden, dass ein Bild damit bearbeitet wird.
Konkret soll dieses in der Helligkeit und im Kontrast angepasst, gespiegelt und in ein Graubild \"uberf\"uhrt werden.
Zudem soll eine Kantendetektion angewendet werden.
Der Hostcode ist dabei zum gr\"o\ss ten Teil vorgegeben, der Devicecode muss im Laufe des Versuchs angepasst werden.

\section{Erprobung}

\subsection{Bild kopieren}
Diese Aufgabe diente als Einf\"uhrung, um mit dem Programmiermodell vertraut zu werden.
Au\ss erdem war damit Code gegeben, auf dem in den anderen Aufgaben aufgebaut werden konnte.
Die Kompilierung und die Ausf\"uhrung liefen wie erwartet.

\subsection{Helligkeit/Kontrast anpassen}
Die Ver\"anderung von Kontrast und Helligkeit eines Bildpunktes ist formal eine lineare Abbildung.
Der Kontrast l\"asst sich anpassen, indem der Farbwert eines Bildpunktes mit einem Faktor multipliziert wird.
Die Helligkeit wird per Addition angepasst.
Ver\"andert werden dabei nur die RGB-Werte, der Alphakanal bleibt unber\"uhrt.
Nach der Berechnung muss beachtet werden, dass die Bildpunkte vom Typ \texttt{unsigned char} sind, was bedeutet, dass sie Werte zwischen 0 und 255 annehmen.
Aus diesem Grund muss man kleinere oder gr\"o\ss ere Werte entsprechend anpassen.
Insgesamt haben wir den folgenden Kernel verwendet:

\begin{lstlisting}[caption=linearTransformKernel,label=a2.2]
int adrIn=(i+j*width)*4;
int adrOut=adrIn;
unsigned char r,g,b,a;
float r_new, g_new, b_new;
        
r = img_in[adrIn+0];
g = img_in[adrIn+1];
b = img_in[adrIn+2];
a = img_in[adrIn+3];
        
r_new = alpha*r + beta;
r_new = r_new < 0?     0 : r_new;
r_new = r_new > 255? 255 : r_new;
        
g_new = alpha*g + beta;
g_new = g_new < 0?     0 : g_new;
g_new = g_new > 255? 255 : g_new;
        
b_new = alpha*b + beta;
b_new = b_new < 0?     0 : b_new;
b_new = b_new > 255? 255 : b_new;
        
img_out[adrOut+0] = (unsigned char)r_new;
img_out[adrOut+1] = (unsigned char)g_new;
img_out[adrOut+2] = (unsigned char)b_new;
img_out[adrOut+3] = a;

\end{lstlisting}

In Listing \ref{a2.2} wurde lediglich der Ausschnitt innerhalb der if-Abfrage des Indexes aufgef\"uhrt, da der \"au\ss ere Teil ohnehin immer gleich ist.
Die eigentliche Berechnung findet innerhalb der Zeilen 11 bis 21 statt.
Zuerst wird die lineare Abbildung angewendet, dann werden die Werte auf das Intervall $[1,255]$ beschr\"ankt.
Abschlie\ss end ist ein Cast nach \texttt{unsigned char} n\"otig.

\subsection{Bild spiegeln}
Nun sollte die linke Bildseite auf die rechte gespiegelt werden.
In dieser Hinsicht unterscheidet sich diese Aufgabe kaum von der ersten, es muss lediglich die Inputadresse angepasst werden.
Die Formeln daf\"ur waren bereits in der Versuchsvorbereitung gegeben, konkret handelt es sich um die Formeln (1) und (3).
Kombiniert man beide, ergibt sich die Berechnung, die in Listung \ref{a2.3} verwendet wird:

\begin{lstlisting}[caption=mirrorKernel,label=a2.3]
int adrIn;
int adrOut=(i+j*width)*4;
unsigned char r,g,b,a;

if (i < width/2) {
    adrIn=adrOut;
} else {
    adrIn=(width-i+j*width)*4;
}

r = img_in[adrIn+0];
g = img_in[adrIn+1];
b = img_in[adrIn+2];
a = img_in[adrIn+3];

img_out[adrOut+0] = r;
img_out[adrOut+1] = g;
img_out[adrOut+2] = b;
img_out[adrOut+3] = a;

\end{lstlisting}

Die betroffene Zeile 8 l\"asst die jeweilige Zeile im Bild (\texttt{j*width}) unver\"andert und w\"ahlt die Spalte von rechts nach links aus, wodurch der Spiegelungseffekt entsteht.
Der Faktor 4 ist notwendig, da jeder Bildpunkt aus 4 Werten (RGBA) besteht.

\subsection{Graubild erstellen}
In dieser Aufgabe sollte ein Graubild basierend auf dem Inputbild erstellt werden.
Mathematisch wird dabei jeder Kanal eines Bildpunkts im Ergebnisbild auf den Durchschnitt der Kanalwerte des Bildpunktes im Ausgangsbild gesetzt.
Dabei bleibt der Alphakanal wieder einmal unber\"uhrt.
Listing \ref{a2.4} zeigt den verwendeten Kernel, bei dem sich lediglich die Zeilen 11 bis 15 von dem Kopierkernel unterscheiden. 

\begin{lstlisting}[caption=bwKernel,label=a2.4]
int adrIn=(i+j*width)*4;
int adrOut=adrIn;
unsigned char r,g,b,a;
unsigned char bw;

r = img_in[adrIn+0];
g = img_in[adrIn+1];
b = img_in[adrIn+2];
a = img_in[adrIn+3];

bw = (r+g+b)/3;

img_out[adrOut+0] = bw;
img_out[adrOut+1] = bw;
img_out[adrOut+2] = bw;
img_out[adrOut+3] = a;
\end{lstlisting}

Insgesamt war diese Aufgabe als Vorbereitung auf die n\"achste zu sehen, welche ein Graubild als Ausgangsbild ben\"otigt.

\subsection{Kantendetektion mit Sobel-Filter}
Eine detaillierte Erl\"auterung der Funktionsweise des Sobel-Filters findet sich in Abschnitt 1.1.5 der Versuchsbeschreibung.
Zusammengefasst werden zwei mit Sobel-Operatoren gewichtete Summen $g_x$ und $g_y$ von einem Bildpunkt und seinen 8 Nachbarn gebildet.
In einem zweiten Schritt wird die 2-Norm des Vektors ($g_x g_y$) als Grauwert in das Ergebnisbild geschrieben, das hei\ss t derselbe Wert in jeden Farbkanal.
Unsere Implementierung ist in Listing \ref{a2.5} zu sehen.

\begin{lstlisting}[caption=sobelKernel,label=a2.5]
int adrIn;
int adrOut=(i+j*width)*4;
int i_new, j_new, bw;
unsigned char r,a;    // is enough since r=g=b in a grayscale picture

float gX = 0.0f, gY = 0.0f;

for (int k = -1; k <= 1; k++) {
    for (int l = -1; l <= 1; l++) {
        i_new = i+k;
        j_new = j+l;

        if (i_new < 0 || i_new > width || j_new < 0 || j_new > height) {
            r = 0;
        } else {
            adrIn=(i_new+j_new*width)*4;
            r = img_in[adrIn];
        }

        gX += SX[1+k][1+l] * r;
        gY += SY[1+k][1+l] * r;
    }
}

adrIn = adrOut;
a = img_in[adrIn+3];

bw = sqrtf (gX*gX + gY*gY);

img_out[adrOut+0] = bw;
img_out[adrOut+1] = bw;
img_out[adrOut+2] = bw;
img_out[adrOut+3] = a;
\end{lstlisting}

Die Summen $g_x$ und $g_y$ werden durch die for-Schleifen in den Zeilen 8 bis 23 realisiert.
Die Variablen \texttt{i\_new} und \texttt{j\_new} sind dabei die Koordinaten des Nachbarn, der gerade getrachtet wird.
Falls sie au\ss erhalb des Bildes liegen sollten, wird ein schwarzer Bildpunkt als Berechnungsgrundlage gew\"alht (Zeilen 13 und 14).
Anderenfalls muss die Adresse im Bildarray berechnet und der jeweilige Farbwert geladen werden.
Es reicht an dieser Stelle vollkommen aus, nur einen Kanal zu laden, da das Ausgangsbild ein Graubild ist, welches die selben Farbwerte in allen Kan\"alen hat.
Wir haben zur leichteren Adressierung den Rotkanal gew\"ahlt.
In den Zeilen 20 und 21 wird dann die gewichtete Summation und in Zeile 28 wird anschlie\ss end die Normberechnung vorgenommen.
Bevor der Output geschrieben werden kann, muss der Wert des Alphakanals geladen werden, was in Zeile 26 geschieht.
Das Ergebnis des Sobel-Filters ist ein Graubild, weshalb wieder jeder Kanal denselben Wert erh\"alt.

Im zweiten Teil dieser Aufgabe musste noch der Host-Code geschrieben werden.
Dieser besitzt gro\ss e \"Ahnlichkeiten zu den \"ubrigen Host-Codes, jedoch gibt es den Unterschied, dass der Graubild- und der Sobel-Kernel nacheinander auf der GPU berechnet werden sollen, ohne dass das Zwischenergebnis auf die CPU kopiert werden soll.
Dazu musste ein Zwischenspeicher auf der GPU reserviert werden, der als Ausgabe f\"ur den bwKernel und als Input f\"ur den sobelKernel verwendet wird.
Nach der Verwendung muss auch dieser Speicher freigegeben werden.
Ein Teil der Aufgabe war, dass der Code aus \texttt{bwCuda()} kopiert werden soll.
In Listing \ref{a2.5host} sind deshalb die Zeilen mit einem Kommentar gekennzeichnet, die wir ver\"andert bzw. hinzugef\"ugt haben.

\begin{lstlisting}[caption=sobelCuda,label=a2.5host,breaklines=true]
unsigned char *img_in_dev, *img_out_dev, *img_bw_dev; //Zwischenspeicher *img_bw_dev deklariert
int size=width*height*4;
cudaMalloc((void**)&img_in_dev,size*sizeof(unsigned char));
cudaMalloc((void**)&img_out_dev,size*sizeof(unsigned char));
cudaMalloc((void**)&img_bw_dev,size*sizeof(unsigned char)); // Zwischenspeicher reserviert
dim3 threads(16,16);
dim3 grid(width/threads.x+1,height/threads.y+1);
cudaMemcpy(img_in_dev,img_in,size*sizeof(unsigned char),cudaMemcpyHostToDevice); // Zwischenspeicher als Ausgabeziel
bwKernel<<<grid,threads>>>(img_in_dev,img_bw_dev,width,height); // Zwischenspeicher als Eingabe
sobelKernel<<<grid,threads>>>(img_bw_dev,img_out_dev,width,height);
cudaMemcpy(img_out,img_out_dev,size*sizeof(unsigned char),cudaMemcpyDeviceToHost);
cudaFree(img_in_dev);
cudaFree(img_out_dev);
cudaFree(img_bw_dev); // Freigeben des Zwischenspeicher
\end{lstlisting}

\section{Schlussfolgerungen}

\end{document}


