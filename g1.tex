
\documentclass[12pt,a4paper]{scrartcl}

\usepackage{amsmath}
\usepackage{fancyhdr}
\usepackage{times}
\usepackage[T1]{fontenc} 
\usepackage[latin1]{inputenc} 
\usepackage[ngerman]{babel}
\usepackage[right=3cm]{geometry}
\usepackage{listings} % for code
\lstset{language=C, numbers=left}

\parindent=0pt
\parskip=1em plus 2pt minus 1pt

\topskip-2cm
\textheight25cm
\pagestyle{fancy}
\fancyhead{}
\fancyfoot{}
\rhead{\thepage}

\begin{document}
\thispagestyle{empty}

\begin{center}
  \LARGE
  Praktische \"Ubungen - \\
  Experimentelle Hardwareprojekte \\
  \bigskip
  \Large 
  Versuchsprotokoll
\end{center}

\vspace{1em}
Versuch: G1 - Parallelrechner - Grafikkarten - Implementierung

Versuchsdatum und -zeit: 30. Mai 2018, 14 - 17 Uhr

Betreuer: Ralf Seidler

\vspace{1em}
Name, Studiengang, Mat.-Nr.: Alexander K\"uhnle, B.Sc. Informatik, 165692

Email: alexander.kuehnle@uni-jena.de

\vspace{1em}
Name, Studiengang, Mat.-Nr.: Mark Umnus, B.Sc. Informatik, 167419

Email: mark.umnus@uni-jena.de

\vspace*{1cm}
%% \mbox{}\\
\hrule
\vspace*{1cm}
{\Large  Eigenst\"andigkeitserkl\"arung }
 
Hiermit versichern wir, dass wir das Protokoll selbstst\"andig verfasst
und keine anderen Quellen und Hilfsmittel als die angegebenen benutzt 
haben. Im Falle einer Zuwiderhandlung erkennen wir an, dass unser Protokoll 
als nicht bestanden bewertet wird und damit das Modul ``Experimentelle 
Hardwareprojekte'' als nicht bestanden bewertet wird. \\
Dar\"uberhinaus ist uns klar, dass jede Zuwiderhandlung ausnahmslos dem 
Rechtsamt der FSU gemeldet wird, woraus weitere Konsequenzen resultieren 
k\"onnen. \\

Unterschriften: \\ 
\hspace*{4cm} ........................................ 
\hspace{2cm} ........................................  \\

\hrule

\vspace*{0.3cm}
\textbf{Vom Betreuer auszuf\"ullen:}

Vorbereitung/Kolloquium:

Durchf\"uhrung:

Protokoll:

Gesamtbewertung:
\clearpage


% Hier geht das Protokoll los...

\section{Vorbereitung}


\section{Vorgehensweise}
Dazu soll in diesem Versuch eine Grafikkarte so programmiert werden, dass ein Bild damit bearbeitet wird.
Konkret soll dieses in der Helligkeit und im Kontrast angepasst, gespiegelt und in ein Graubild \"uberf\"uhrt werden.
Zudem soll eine Kantendetektion angewendet werden.
Der Hostcode ist dabei zum gr\"o\ss ten Teil vorgegeben, der Devicecode muss im Laufe des Versuchs angepasst werden.

\section{Erprobung}

\subsection{Bild kopieren}
Diese Aufgabe diente als Einf\"uhrung, um mit dem Programmiermodell vertraut zu werden.
Au\ss erdem war damit Code
Die Kompilierung und die Ausf\"uhrung liefen wie erwartet.


\section{Schlussfolgerungen}

\end{document}


