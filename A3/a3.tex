
\documentclass[12pt,a4paper]{scrartcl}

\usepackage{amsmath}
\usepackage{fancyhdr}
\usepackage{times}
\usepackage[T1]{fontenc} 
\usepackage[latin1]{inputenc} 
\usepackage[ngerman]{babel}
\usepackage[right=3cm]{geometry}
\usepackage{listings} % for code
\lstset{frame=single,numbers=left,breaklines=true}

\usepackage{tikz}

\parindent=0pt
\parskip=1em plus 2pt minus 1pt

\topskip-2cm
\textheight25cm
\pagestyle{fancy}
\fancyhead{}
\fancyfoot{}
\rhead{\thepage}

\begin{document}
\thispagestyle{empty}

\begin{center}
  \LARGE
  Praktische \"Ubungen - \\
  Experimentelle Hardwareprojekte \\
  \bigskip
  \Large 
  Versuchsprotokoll
\end{center}

\vspace{1em}
Versuch: A3 - Befehlssatzerweiterung und Test eines RISC-Prozessors

Versuchsdatum und -zeit: 21. Juni 2018, 10 Uhr - 13 Uhr

Betreuer: Andreas Reinsch

\vspace{1em}
Name, Studiengang, Mat.-Nr.: Alexander K\"uhnle, B.Sc. Informatik, 165692

Email: alexander.kuehnle@uni-jena.de

\vspace{1em}
Name, Studiengang, Mat.-Nr.: Mark Umnus, B.Sc. Informatik, 167419

Email: mark.umnus@uni-jena.de

\vspace*{1cm}
%% \mbox{}\\
\hrule
\vspace*{1cm}
{\Large  Eigenst\"andigkeitserkl\"arung }
 
Hiermit versichern wir, dass wir das Protokoll selbstst\"andig verfasst
und keine anderen Quellen und Hilfsmittel als die angegebenen benutzt 
haben. Im Falle einer Zuwiderhandlung erkennen wir an, dass unser Protokoll 
als nicht bestanden bewertet wird und damit das Modul ``Experimentelle 
Hardwareprojekte'' als nicht bestanden bewertet wird. \\
Dar\"uberhinaus ist uns klar, dass jede Zuwiderhandlung ausnahmslos dem 
Rechtsamt der FSU gemeldet wird, woraus weitere Konsequenzen resultieren 
k\"onnen. \\

Unterschriften: \\ 
\hspace*{4cm} ........................................ 
\hspace{2cm} ........................................  \\

\hrule

\vspace*{0.3cm}
\textbf{Vom Betreuer auszuf\"ullen:}

Vorbereitung/Kolloquium:

Durchf\"uhrung:

Protokoll:

Gesamtbewertung:
\clearpage


% Hier geht das Protokoll los...

\section{Vorbereitung}
Die wichtigste Grundlage zur Vorbereitung auf diesen Versuch war die Beschreibung des DLXJ-Prozessors, mit deren Hilfe wir uns verschiedene Konzepte klargemacht haben.

\subsection*{Befehlsstruktur}
Zuerst haben wir die Hexadezimaldarstellung von drei Kommandos betrachtet und diese decodiert.
Der erste Schritt dabei ist es, die Hex-Darstellung in eine Bin\"ardarstellung umzuwandeln, da die Argumente der Befehle nicht in Tetraden organisiert sind.
Der erste Befehl sieht dann folgenderma\ss en aus:
\begin{align*}
\text{0x50010003} & \equiv \underbrace{010100}_{OP-Code}\underbrace{00000}_{Rs1}\underbrace{00001}_{Rd}\underbrace{0000000000000011}_{Immediate} \\
\text{0x0021100C} & \equiv \underbrace{000000}_{OP-Code}\underbrace{00001}_{Rs1}\underbrace{00001}_{Rs2}\underbrace{00010}_{Rd}\underbrace{00000}_{unused}\underbrace{001100}_{rr\_func} \\
\text{0x00411809} & \equiv \underbrace{000000}_{OP-Code}\underbrace{00010}_{Rs1}\underbrace{00001}_{Rs2}\underbrace{00011}_{Rd}\underbrace{00000}_{unused}\underbrace{001001}_{rr\_func}
\end{align*}

Die Bin\"arbefehle wurden hier direkt interpretiert.
Dies geschah auf Grundlage des OP-Codes, der bei den letzten beiden Befehlen bestimmt, dass es sich um R-Typ-Befehle handelt, w\"ahrend der erste ein I-Typ-Befehl ist.
Auf die genaue Unterscheidung wird sp\"ater noch einmal eingegangen.
Zun\"achst werden die Befehle in Tabelle \ref{tab:vorbinstr} zur Veranschaulichung in Assemblercode zur\"uck\"ubersetzt und dann ausgewertet.

\begin{table}[h]
    \centering
    \begin{tabular}{l|l|l}
    \hline
    Befehl      & Assemblercode     & Registerinhalte nach Befehl           \\
    \hline
    0x50010003  & add.i r1, r0, 3   & R1 <- R0 + 3 = 0 + 3 = 3              \\
    0x0021100C  & sll   r2, r1, r1  & R2 <- R1 << R1 = 3 << 3 = 24          \\
    0x00411809  & or    r3, r2, r1  & R3 <- R2 $\vee$ R1 = 24 $\vee$ 3 = 27 \\
    \hline
    \end{tabular}
    \caption{Deassemblierte Befehle mit Registerupdates}
    \label{tab:vorbinstr}
\end{table}

\subsection*{R- und I-Befehlstypen}
Um in den n\"achsten Aufgaben die neuen Befehle implementieren zu k\"onnen, sollen hier kurz die Unterschiede zwischen R- und I-Typ-Befehlen erkl\"art werden.
Der erste Unterschied liegt im Befehlscodeformat.
W\"ahrend R-Typen zwei Quellregister und eine \texttt{rr\_func} angeben sowie einen Teil des Befehlscodes gar nicht nutzen, geben I-Typen ein Quellregister und eine 16-Bit-Konstante an.
Diese Unterscheidung ist jedoch f\"ur das Steuerwerk und auch f\"ur die ALU irrelevant.
Lediglich der Decoder 3 entscheidet auf dieser Basis, welcher Wert auf den S2-Bus gelegt werden soll - die Konstante oder ein Registerinhalt.
F\"ur die zu implementierenden Befehle bedeutet dies, dass jeweils nur ein neuer Zustand angelegt werden muss.

\subsection*{Erweiterung des Steuerwerks}
Wir haben also 5 neue Zust\"ande definiert: \texttt{add}, \texttt{and\_1}, \texttt{or\_1}, \texttt{sll\_1} und \texttt{srl\_1}, zu sehen auf Abbildung \ref{fig:fsm}.
Sie alle k\"onnen durch die Bedingung $(op\_rr\_func \wedge rr\_func\_***) \vee op\_***\_i$ vom Zustand \texttt{dec\_pcinc4\_ab} aus erreicht werden, wobei *** durch die jeweilige Befehlskennung zu ersetzen ist.
Ebenso f\"uhren alle neuen Zust\"ande nach der Abarbeitung des jeweiligen Befehls in den Zustand \texttt{wr\_back}.
Als Berechnung haben wir f\"ur einen Befehl \texttt{$C \leftarrow A \circ Y$} eingetragen, wobei A der Inhalt des Registers A ist und Y der Inhalt des Registers B oder ein Immediate-Wert, je nachdem, welcher Typ des Befehls vorliegt.
$\circ$ bezeichnet die jeweilige Operation.
Auf diese Weise haben wir auch den \texttt{sub}-Zustand angepasst.

\begin{figure}
    \begin{tikzpicture}[scale=3] 
        \path  (3, 2)  node[draw] (DEC) {dec\_pcinc4\_ab} 
               (1, 1.5)  node[draw,align=center] (ADD) {add \\ $C \leftarrow A + Y$}
               (2, 1.5)  node[draw,align=center] (AND) {and\_1 \\ $C \leftarrow A \wedge Y$}
               (3, 1.5)  node[draw,align=center] (OR)  {or\_1 \\ $C \leftarrow A \vee Y$}
               (4, 1.5)  node[draw,align=center] (SRL) {srl\_1 \\ $C \leftarrow A << Y$}
               (5, 1.5)  node[draw,align=center] (SLL) {sll\_1 \\ $C \leftarrow A >> Y$}
               (3, 1)  node[draw,align=center] (WRB) {wr\_back \\ $RD \leftarrow C$};
        \draw[->,gray] (DEC) --  (ADD.north);
        \draw[->,gray] (DEC) --  (AND.north);
        \draw[->,gray] (DEC) --  node[black] {$(op\_rr\_func \wedge rr\_func\_***) \vee op\_***\_i$} (OR.north);
        \draw[->,gray] (DEC) --  (SRL.north);
        \draw[->,gray] (DEC) --  (SLL.north);
        \draw[->] (ADD.south) -- (WRB);
        \draw[->] (AND.south) -- (WRB);
        \draw[->] (OR.south) -- (WRB);
        \draw[->] (SRL.south) -- (WRB);
        \draw[->] (SLL.south) -- (WRB);
        \draw (0.5, 0.8) -- (0.5, 2.2) -- (5.5, 2.2) -- (5.5, 0.8) -- (0.5, 0.8);
    \end{tikzpicture}
    \label{fig:fsm}
    \caption{Neue Zust\"ande der FSM des Steuerwerks}
\end{figure}

In all diesen neuen Zust\"anden (und \texttt{sub}) m\"ussen als Datenquellen die Signale \texttt{s1\_a} und \texttt{s2\_y} gew\"ahlt werden.
Als Datensenke wird jeweils \texttt{dest\_c} angesteuert.

\subsection*{VHDL-Erweiterung}
Im n\"achsten Schritt haben wir die gegebenen VHDL-Beschreibungen des Decoders 1, der Zustands\"uberf\"uhrungs- und der Ausgabefunktion erg\"anzt.
S\"amtliche Namen waren schon vordefiniert und mussten lediglich richtig eingesetzt werden.

\subsubsection*{Decoder 1}
Der Decoder 1 extrahiert die ALU-Operation aus dem Befehl.
Dazu gibt es zwei \texttt{case}-Statements:
Eines, in dem die letzten 6 Bit als RR-Funktion interpretiert werden und eines, bei dem direkt der OP-Code betrachtet wird.
Diese Konstellation f\"uhrt zu der bei Abbildung \ref{fig:fsm} genannten logischen Formel $(op\_rr\_func \wedge rr\_func\_***) \vee op\_***\_i$.
Unsere konkreten Erg\"anzungen sind in Listing \ref{lst:dec1} zu sehen.
Die Kommentare deuten dabei die umgebenden Codezeilen an.
\lstinputlisting[label=lst:dec1, caption=Erg\"anzungen in \texttt{ir\_decode\_1-behavior.vhd}]{decoder1.vhd}

\subsubsection*{Zustands\"uberf\"uhrungsfunktion}
In der Datei \texttt{fsm\_next-dataflow.vhd} mussten lediglich noch die \"Uberg\"ange von den neuen Zust\"anden zu \texttt{wr\_back} hinzugef\"ugt werden.
In die neuen Zust\"ande gelangt man \"uber folgende Zeilen, die bereits vorgegeben waren:
\begin{lstlisting}
WHEN dec_pcinc4_ab =>
next_state <= dec_1_in;
\end{lstlisting}

Nach dem gleichen Muster haben wir, wie in Listing \ref{lst:zuef} zu sehen, unsere die neu definierten Operationen implementiert:
\lstinputlisting[label=lst:zuef,caption=Erg\"anzungen in \texttt{fsm\_next-dataflow.vhd}]{next_state.vhd}

\subsubsection*{Ergebnisfunktion}
In der Ergebnis- oder Ausgabefunktion wird festgelegt, welche Signale vom Steuerwerk aus aktiviert werden.
F\"ur jeden neuen Zustand m\"ussen alle Steuersignale definiert werden, was viel Platz in diesem Dokuemnt kostet.
Deshalb und wegen der hochgradigen Analogie ist in Listing \ref{lst:resf} lediglich gezeigt, wie wir die Ergebnisfunktion f\"ur den \texttt{add}-Zustand beschrieben haben.
Die vollst\"andige Erg\"anzung ist im Anhang aufgef\"uhrt.
\lstinputlisting[label=lst:resf,caption=Auszug aus den Erg\"anzungen in \texttt{fsm\_output-dataflow.vhd},firstline=11,lastline=20]{output_state.vhd}

Wir m\"ochten hier noch einmal auf die Verwendung von \texttt{s2\_Y} hinweisen, die dem Decoder 3 die Wahl des Wertes f\"r den S2-Bus \"uberl\"asst.
Besondere Vorsicht erfordert auch das Signal \texttt{immed\_sel}, bei dem entschieden wird, ob der Wert der Konstante -- sofern es eine gibt -- vorzeichenerweitert oder mit Nullen aufgef\"ullt an die ALU weitergegeben wird.

Nachdem der Prozessor um die Zust\"ande f\"ur neue arithmetisch-logische Operationen erweitert wurde, haben wir als n\"achstes ein Assemblerprogramm geschrieben, um die Funktionen zu testen.

\subsection*{Assemblerprogramm}
Dieses Testprogramm besteht aus einer Aneinanderreihung der neuen Befehle.
Um den Code kleinzuhalten haben wir darauf verzichtet, die Ergebnisse auf dem Display auszugeben.
F\"ur die bessere \"Ubersicht \"uber die Resultate haben wir jedem Ergebnis ein eigenes Register zugewiesen.
So ist es im sp\"ateren Register Dump m\"oglich, das Programm auf Richtigkeit zu pr\"ufen.
Der Quellcode ist in Listing \ref{lst:test} dargestellt.
\lstinputlisting[label=lst:test, caption=Testprogramm]{test.asm}

Auch nach der letzten Anweisung eines Programms wird der Befehlsz\"ahler des DLXJ-Prozessors weiter inkrementiert, es werden neue Speicherinhalte geladen und der Prozessor versucht, diese auszuf\"uhren.
Um ungewolltes Verhalten zu verhindern, haben wir unser Programm mit einer Endlosschleife abgeschlossen.
Im g\"unstigsten Fall haben die Speicherinhalte ein ung\"ultiges Format und der Prozessor wechselt in den Error-State.
Dann ist es m\"oglich, den Prozessor per reset-Signal zur\"uckzusetzen und Befehl f\"ur Befehl den Code abarbeiten, bis man die offene Stelle findet.
Es k\"onnten aber auch beliebige Befehle im Speicher stehen die ebenso beliebige Effekte h\"atten.
Das sollte auf jeden Fall ausgeschlossen werden.
Eine Endlosschleife verhindert effektiv, dass der Befehlsz\"ahler hochgez\"ahlt wird, da er immer auf das Label \texttt{end} zur\"uckgesetzt wird.

\section{Vorgehensweise}
In diesem Versuch wurden die zus\"atzlichen Befehle implementiert.
Anschlie\ss end wurde die Abarbeitung unseres Assemblerprogramms durch den DLXJ-Prozessor simuliert und auf dem Experimentalsystem mit Debugger getestet.

\section{Erprobung}
\subsection*{Implementierung}
Dank unserer ausf\"uhrlichen Vorbereitung konnten wir unsere Aufgaben im Praktikum z\"ugig erledigen.
Zuerst haben wir die VHDL-Dateien des Decoders 1, der Ergebnis- und der \"Uberf\"uhrungsfunktion angepasst.
Dazu haben wir unseren Code an die bereits markierten Stellen kopiert.
Das \"Ubersetzen mit dem Programm \texttt{dlxj\_analyze\_for\_funcsim} verlief fehlerlos.

\subsection*{Simulation des Assemblerprogramms}
Im Anschluss haben wir unser vorbereitetes Assemblerprogramm in den /asm-Ordner kopiert und mit \texttt{dlxj\_assemble\_program test.asm} \"ubersetzen lassen.
Mit dem Programm \texttt{dlxj\_simulate\_function} konnten wir die Simulation des Assemblerprogramms starten.
Hierbei handelte es sich um das gleiche Programm, das auch schon in den Versuchen A1 und A2 zum Einsatz kam.
Wir haben beobachtet, wie sich die Steuerleitungen \texttt{/b\_en\_out} und \texttt{/immed\_en\_out2} im Zeitverlauf verhalten.
Erwartungsgem\"a\ss\  sind niemals beide gleichzeitig auf 0 gesetzt (also die Leitungen aktiv).
Bei R-Typ-Befehlen ist \texttt{/b\_en\_out} 0, sodass der Inhalt des Registers B des Datenpfades auf dem S2-Bus liegt.
Bei I-Typ-Befehlen wird der Immediate-Wert auf den S2-Bus gelegt, indem \texttt{/immed\_en\_out2} den Wert 0 erh\"alt.
Des Weiteren haben wir die Zust\"ande des Steuerwerks protokolliert, die korrekterweise dem Schema \texttt{FETCH} -> \texttt{DEC\_PCINC4\_AB} -> Befehl -> \texttt{WR\_BACK} folgten.
Die Registerbelegungen im Befehlsverlauf haben wir gedruckt; sie sind im Anhang xy zu finden, wobei die Ergebnisse jeweils markiert sind.
Es traten keine Abweichungen von den Vor\"uberlegungen auf.

\subsection*{Test des Assemblerprogramms auf dem Experimentalsystem}
Zuletzt haben wir unsere angepasste Prozessorbeschreibung mittels \texttt{dlxj\_make\_fpga} und \texttt{dlxj\_configure\_fpga} implementiert und auf das FPGA \"ubertragen.
Mit dem Programm \texttt{dlxj\_program\_rom} lie\ss\  sich der assemblierte Testcode in den Programmspeicher des DLXJ-Prozessors schreiben.
Nach diesem Vorgang haben wir den Debugger gestartet und das Program mehrmals laufen lassen: sowohl mit Taktperioden- (Kommando \#) als auch mit Befehlsschrittweite (Kommando \texttt{ni}).
Auch hier konnten wir am Ende die Ergebnisse aus der Vorbereitung in den Registern sehen.
Ein Memory Dump vom Ende der Bearbeitung liegt im Anhang xy bei.
Auf diesem kann man auch erkennen, dass die Endlosschleife am Ende des Programms funktioniert:
Im Instruction Register liegt der Befehl 0x33ff\_fffc.
\begin{align*}
\text{0x33fffffc} \equiv \underbrace{\text{0011 00}}_{\text{OP-Code J}}\underbrace{\text{11 1111 1111 1111 1111 1111 1100}}_{\text{Immediate -4}}
\end{align*}
Deassembliert ist dies ein Sprungbefehl, der eine Konstante zu dem PC addiert, die in Zweierkomplementdarstellung -4 entspricht.
Effektiv bedeutet dies, dass sich der PC nicht \"andert; im n\"achsten Durchlauf wird derselbe Befehl in das Instruction Register geladen.

\section{Schlussfolgerungen}
In diesem Versuch haben wir gelernt, wie ein RISC-Prozessor im Detail funktioniert und wie man ihn debuggen kann.
Au\ss erdem hat uns dieser Versuch gezeigt, wie hilfreich eine gute Vorbereitung sein kann.

\section{Anhang}
\lstinputlisting[caption=Erg\"anzungen in \texttt{fsm\_output-dataflow.vhd}]{output_state.vhd}

\end{document}
