
\documentclass[12pt,a4paper]{scrartcl}

\usepackage{fancyhdr}
\usepackage{times}
\usepackage[T1]{fontenc} 
\usepackage[latin1]{inputenc} 
\usepackage[ngerman]{babel}
\usepackage[right=3cm]{geometry}
\usepackage{wasysym} %for moon symbols
\parindent=0pt
\parskip=1em plus 2pt minus 1pt

\topskip-2cm
\textheight25cm
\pagestyle{fancy}
\fancyhead{}
\fancyfoot{}
\rhead{\thepage}

\begin{document}
\thispagestyle{empty}

\begin{center}
  \LARGE
  Praktische \"Ubungen - \\
  Experimentelle Hardwareprojekte \\
  \bigskip
  \Large 
  Versuchsprotokoll
\end{center}

\vspace{1em}
Versuch: A1 - CAE-Werkzeuge

Versuchsdatum und -zeit: 26. August 2018, 10 - 13 Uhr 

Betreuer: Volker D\"orsing

\vspace{1em}
Name, Studiengang, Mat.-Nr.: Alexander K\"uhnle, B.Sc. Informatik, 165692

Email: alexander.kuehnle@uni-jena.de

\vspace{1em}
Name, Studiengang, Mat.-Nr.: Mark Umnus, B.Sc. Informatik, 167419

Email: mark.umnus@uni-jena.de

\vspace*{1cm}
%% \mbox{}\\
\hrule
\vspace*{1cm}
{\Large  Eigenst\"andigkeitserkl\"arung }
 
Hiermit versichern wir, dass wir das Protokoll selbstst\"andig verfasst
und keine anderen Quellen und Hilfsmittel als die angegebenen benutzt 
haben. Im Falle einer Zuwiderhandlung erkennen wir an, dass unser Protokoll 
als nicht bestanden bewertet wird und damit das Modul ``Experimentelle 
Hardwareprojekte'' als nicht bestanden bewertet wird. \\
Dar\"uberhinaus ist uns klar, dass jede Zuwiderhandlung ausnahmslos dem 
Rechtsamt der FSU gemeldet wird, woraus weitere Konsequenzen resultieren 
k�nnen. \\

Unterschriften: \\ 
\hspace*{4cm} ........................................ 
\hspace{2cm} ........................................  \\

\hrule

\vspace*{0.3cm}
\textbf{Vom Betreuer auszuf�llen:}

Vorbereitung/Kolloquium:

Durchf�hrung:

Protokoll:

Gesamtbewertung:
\clearpage


% Hier geht das Protokoll los...

\section{Vorbereitung}

Die A-Versuchsreihe soll die Kenntnisse der Vorlesungen \emph{Technische Informatik} und \emph{Rechnerstrukturen} anhand der RISC-Prozessorarchitektur vertiefen.
Dazu besch\"aftigt sich der Versuch A1 mit einer Grundstruktur, dem D-Latch.
Das Ziel ist es, den Arbeitsablauf von der Entwicklung einer Schaltung \"uber funktionale Simulationen, Synthese, Implementierung, zeitbehaftete Simulationen bis zur Erprobung kennenzulernen.
Das D-Latch sollte dabei mit 4-Bit-Ein- und Tristateausg\"angen, einem Takteingang (CLK) sowie zwei Steuerleitungen f\"ur das Aktivieren und Deaktivieren des Taktes (CLK\_EN) und des Ausganges (out\_en) realisiert werden.


\section{Vorgehensweise}

Uns lagen f\"ur den Versuch bereits zahlreiche Dateien vor, namentlich 
\texttt{*.vhd} (die VHDL-Quellen),
\texttt{tb.vhd} (die Testumgebung),
\texttt{xc3s700an.ucf} (Vorgaben f\"ur die Implementierung, Anordnung der Anschl\"usse), 
\texttt{synopsys\_sim.setup} (Initialisierungsdatei f\"ur Synopsys),
\texttt{an.sh} (zur Analyse von tb.vhd),
\texttt{fsim.sh} (zur funktionellen Simulation),
\texttt{imp.sh} (zur Synthese und Implementierung),
\texttt{tsim.sh} (zur zeitbehafteten Simulation)

Insbesondere war eine \texttt{dlatch.vhd} gegeben, die ein D-Latch mit den Eing\"angen DATA\_IN(3:0) (4 Dateninputsignale) und CLK (Takt), den Ausg\"angen DATA\_OUT(3:0) (4 Datenausg\"ange) sowie den Steuerleitungen CLK\_EN (Aktivieren und Deaktivieren des Takes) und OUT\_EN (Hochohmigschalten der Ausg\"ange falls 1) beschreibt.

Zun\"achst musste im Versuch die \texttt{tb.vhd} angepasst werden, um in den Simulationen die verschiedenen Funktionen des D-Latch zeigen und um die Eingabesignale auf ihre Funktion \"uberpr\"ufen zu k\"onnen.
Anschlie\ss end wurde die funktionale Simulation durchgef\"uhrt und die Ergebnisse ausgedruckt.
Im n\"achsten Schritt wurde dann mittels des gegebenen Skriptes \texttt{imp.sh} die Synthese und Implementierung vorgenommen.
Danach wurde das Ergebnis im FPGA-Editor analysiert.
Es folgte die zeitbehaftete Simulation, die im Vergleich zur funktionalen Simulation Zeitunterschiede ber\"ucksichtigt, die durch die Implementierung bedingt sind.
Abschlie\ss end wurde das FPGA-System konfiguriert, sodass es mithilfe des Programmes \texttt{tst} getestet werden konnte.

\emph{Anmerkung}:
In diesem Protokoll werden des \"Ofteren die Befehle des \texttt{tst}-Programmes zur Beschreibung unseres Vorgehens verwendet. Diese sind in den Versuchsunterlagen auf Seite 9 aufgelistet.

\section{Erprobung}

\subsection{Funktionale Simulation}
Hier haben wir die Funktionen \emph{Transparenz}, \emph{Speicherung}, \emph{Ausgabeunterdr\"uckung} und \emph{Taktunterdr\"ckung} gezeigt.
Einen \"Uberblick \"uber die Funktionen bietet Tabelle \ref{wwerte}.
In dieser bezeichnet MEMORY den in dem Moment an DATA\_IN anliegenden Wert, als $(CLK\_EN \wedge CLK) \equiv 0$ wurde.
Damit gilt, dass die Ausgabe gleich gehalten wird, selbst wenn sich die Eingabe \"andert.
Des weiteren gilt, dass $x,y \in \{0,1\}$ beliebige Werte darstellen.
Der Fall DATA\_OUT = DATA\_IN in der Tabelle impliziert, dass das D-Latch transparent geschalten ist, also Daten vom Eingang direkt an den Ausgang gelangen.
Wenn DATA\_OUT den Wert Z annimmt, bedeuted dies ein Hochohmigschalten dieses Ausganges.

\begin{table}[h]
    \centering
    \begin{tabular}{l|l|l|l|r}
    \hline
    OUT\_EN & CLK\_EN & CLK & DATA\_OUT & Funktion\\
    \hline
    0       & 0       & y   & MEMORY    & Taktunterdr\"uckung\\
    0       & 1       & 0   & MEMORY    & Speicherung\\
    0       & 1       & 1   & DATA\_IN  & Transparenz\\
    1       & x       & y   & Z         & Ausgabeunterdr\"uckung\\
    \hline
    \end{tabular}
    \caption{Wahrheitswertabelle der Eing\"ange mit ihrer Auswirkung auf die Ausg\"ange}
    \label{wwerte}
\end{table}

Mit den vier in Tabelle \ref{wwerte} stehenden Funktionen haben wir die Testbench erweitert und eine funktionelle Simulation durchgef\"uhrt, deren Ergebnis in Anlage 1 zu sehen ist.
Es ergaben sich kurze Verz\"ogerungszeiten zwischen den \"Anderungen der Eing\"ange und der Reaktion an der Ausgabe von jeweils etwa 1ns.
Die genauen Angaben stehen in Tabelle \ref{vergleich}, sind aber eher unerheblich.

\subsection{Synthese und Simulation}
Durch das Skript \texttt{imp.sh} konnten diese beiden Schritte f\"ur uns zusammengefasst werden.
Nach der Ausf\"uhrung konnte das Ergebnis im Xilinx FPGA-Editor betrachtet werden, wobei der anschaulichste Teil in Tabelle \ref{netzliste} dargestellt ist.

\begin{table}[h]
    \centering
    \begin{tabular}{l|l|l}
    \hline
    Typ     & Anzahl  & Beispielortsname \\
    \hline
    IBUF    & 7       & B13              \\
    IOB     & 4       & R20              \\
    BUFGMUX & 1       & BUFGMUX\_X2Y11   \\
    SLICEL  & 2       & SLICE\_X52Y69    \\
    \hline
    \end{tabular}
    \caption{FPGA-Komponenten der Netzliste}
    \label{netzliste}
\end{table}

Im FPGA werden sieben Inputbuffer benutzt, n\"amlich f\"ur die vier Daten-Inputs, f\"ur CLK, CLK\_EN und OUT\_EN.
Die vier I/O-Bl\"ocke werden f\"ur die Umsetzung von DATA\_OUT verwendet.
In ihnen wird jeweils eine Leitung des internen Zustandes des FPGA, d.h. des Inputs seit dem letzten Mal CLK = CLK\_EN = 1, zusammen mit $\neg$OUT\_EN in ein AND-Gate gef\"uhrt und das Ergebnis an ein \texttt{PAD} weitergegeben.
Die Slices verf\"ugen im Inneren \"uber eigene D-Latches, die die eigentliche Funktion in diesem Experiment \"ubernehmen.
Sie sind an zwei Leitungen mit den Bezeichnungen \emph{clk\_en\_c} und \emph{clk\_c} angeschlossen, um entsprechend auf Eingaben reagieren zu k\"onnen.

F\"ur den weiteren Verlauf sind die drei in diesem Schritt entstandenen Dateien \texttt{fpga.bit}, \texttt{fpga.sdf} und \texttt{fpga.vhd} bedeutend.
Erstere enth\"alt die Informationen, die ben\"otigt werden, um das Produkt der Implementierung auf die physische Hardware zu schreiben.
In der letzteren Datei wird der FPGA in VHDL beschrieben, um ihn in der folgenden zeitbehafteten Simulation testen zu k\"onnen.
Die \texttt{fpga.sdf}-Datei liefert die dazu ben\"otigten Zeitinformationen, die u.a. durch die Entfernung der Bauteile voneinander verschieden von der Beschreibung des D-Latch im ersten Schritt und von anderen Implementierungen sein k\"onnen.

\subsection{Zeitsimulation}
In diesem Schritt haben wir die zeitbehaftete Simulation mittels \texttt{tsim.sh} durchgef\"uhrt.
Das Ergebnis ist in Anlage 2 zu sehen.
Am auff\"alligsten ist hier der Zeitunterschied zwischen der funktionalen und der zeitbehafteten Simulation, der in Tabelle \ref{vergleich} zu sehen ist.
Gemessen wurde dabei an der entsprechenden Stelle, die in den Anlagen 1 und 2 markiert ist.

\begin{table}[h]
    \centering
    \begin{tabular}{l|l|l}
    \hline
    Stelle & Verz\"ogerung funktionale Simulation & Verz\"ogerung Zeitsimulation\\
    \hline
    3      & 1.1                                  & 11.3  \\
    4      & 1.1                                  & 10.6  \\
    5      & 1.0                                  & 8.9   \\
    \hline
    \end{tabular}
    \caption{Vergleich der Zeiten zwischen funktionaler und zeitbehafteter Simulation in ns}
    \label{vergleich}
\end{table}

Als weiteren Unterschied zur funktionalen Simulation haben wir festgestellt, dass die Werte an DATA\_OUT bitweise nacheinander ge\"andert werden und nicht gleichzeitig.
Dieses Verhalten erkennt man in Anlage 2 durch nicht so saubere \"Uberg\"ange wie in Anlage 1.
Ein dritter Unterschied ist, dass das D-Latch in der Zeitsimulation weitaus l\"anger zur Initialisierung ben\"otigt, was sich in einem breiteren grauen Bereich zu Beginn des Experimentes an DATA\_OUT ausdr\"uckt.

\subsection{Erprobung der Experimentalschaltung}
Nachdem wir unsere \texttt{fpga.bit} wie in der Anleitung beschrieben auf das Experimentalsystem \"ubertragen haben, konnten wir es mit \texttt{tst} testen.
Unsere Versuchsabfolge aus der Testbench lieferte die erwarteten Ergebnisse.
Das FGPA-System startete dabei wie in den Simulationen so, dass alle Inputs au\ss er DATA\_IN auf 0 gesetzt waren und letzteres auf 0xA.

\newcommand{\off}{\fullmoon}
\newcommand{\on}{\newmoon}

\begin{table}[h]
    \centering
    \caption{Erprobung der Experimentalschaltung mit unserer Versuchsabfolge.
    \on\ bedeutet LED an.}
    \begin{tabular}{l|l|l}
    \hline
    Stelle & \texttt{tst}-Kommado & LED-Anzeige    \\
    \hline
    1      & sce               & \off \off \off \off  \\
    2      & wd 0x6            & \off \off \off \off  \\
    3      & sck               & \off \on  \on  \off \\
    4      & wd 0xc            & \on  \on  \off \off \\
    5      & soe               & \off \off \off \off \\
    6      & rck               & \off \off \off \off \\
    7      & roe               & \on  \on  \off \off  \\
    8      & sck               & \on  \on  \off \off \\
    9      & rce               & \on  \on  \off \off \\
    10     & wd 0xf            & \on  \on  \off \off \\
    11     & sce               & \on  \on  \on  \on  \\
    \hline
    \end{tabular}
    \label{erprobung}
\end{table}

Bei weiteren Versuchseingaben konnte kein unerwartetes Verhalten festgestellt werden.


\section{Schlussfolgerungen}
In diesem Experiment haben wir erfolgreich ein D-Latch von der Beschreibung in VHDL in eine Implementierung auf einem FPGA-System \"uberf\"uhrt.
Dabei haben wir die einzelnen Schritte kennengelernt und verinnerlicht, sodass wir den Prozess in den n\"achsten Experimenten umsetzen k\"onnen.


\end{document}
