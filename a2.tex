\documentclass[12pt,a4paper]{scrartcl}

\usepackage{fancyhdr}
\usepackage{times}
\usepackage[T1]{fontenc}
\usepackage[latin1]{inputenc}
\usepackage[ngerman]{babel}
\usepackage[right=3cm]{geometry}
\usepackage{wasysym} %for moon symbols
\parindent=0pt
\parskip=1em plus 2pt minus 1pt

\topskip-2cm
\textheight25cm
\pagestyle{fancy}
\fancyhead{}
\fancyfoot{}
\rhead{\thepage}

\begin{document}
\thispagestyle{empty}

\begin{center}
  \LARGE
  Praktische \"Ubungen - \\
  Experimentelle Hardwareprojekte \\
  \bigskip
  \Large
  Versuchsprotokoll
\end{center}

\vspace{1em}
Versuch: A2 - Befehlsz\"ahler eines RISC-Prozessors

Versuchsdatum und -zeit: 16. Mai 2018, 14 - 17 Uhr

Betreuer: Volker D\"orsing

\vspace{1em}
Name, Studiengang, Mat.-Nr.: Alexander K\"uhnle, B.Sc. Informatik, 165692

Email: alexander.kuehnle@uni-jena.de

\vspace{1em}
Name, Studiengang, Mat.-Nr.: Mark Umnus, B.Sc. Informatik, 167419

Email: mark.umnus@uni-jena.de

\vspace*{1cm}
%% \mbox{}\\
\hrule
\vspace*{1cm}
{\Large  Eigenst\"andigkeitserkl\"arung }

Hiermit versichern wir, dass wir das Protokoll selbstst\"andig verfasst
und keine anderen Quellen und Hilfsmittel als die angegebenen benutzt
haben. Im Falle einer Zuwiderhandlung erkennen wir an, dass unser Protokoll
als nicht bestanden bewertet wird und damit das Modul ``Experimentelle
Hardwareprojekte'' als nicht bestanden bewertet wird. \\
Dar\"uberhinaus ist uns klar, dass jede Zuwiderhandlung ausnahmslos dem
Rechtsamt der FSU gemeldet wird, woraus weitere Konsequenzen resultieren
k\"onnen. \\

Unterschriften: \\
\hspace*{4cm} ........................................
\hspace{2cm} ........................................  \\

\hrule

\vspace*{0.3cm}
\textbf{Vom Betreuer auszuf\"ullen:}

Vorbereitung/Kolloquium:

Durchf\"uhrung:

Protokoll:

Gesamtbewertung:
\clearpage


% Hier geht das Protokoll los...

\section{Vorbereitung}
In Versuch A2 wird der Befehlsz\"ahler der DLXJ-Prozessorarchitektur n\"aher betrachtet.
Ziel des Versuches ist es, die ben\"otigten Signalfolgen zu verstehen, damit der Befehlsz\"ahler korrekt funktioniert.
Dazu sollten wir in der funktionalen und zeitbehafteten Simulation den Befehlsz\"ahler untersuchen und verstehen, unter welchen Vorraussetzungen
die Funktionen des Befehlsz\"ahlers ausgef\"uhrt werden k\"onnen.

\section{Vorgehensweise}

Uns lagen f\"ur den Versuch mehrere Dateien vor: F\"ur die Testumgebung eine \texttt{tb.vhd} und weitere \texttt{*.vhd} f\"ur die Beschreibung der Komponenten des Befehlsz\"ahlers. Au\ss erdem wurden uns, analog zu Versuch A1, die Skripte \texttt{fsim.sh} und \texttt{tsim.sh} f\"ur die Analysen und \texttt{imp.sh} f\"ur Synthese und Implementierung, zur Verf\"ugung gestellt.

Zuerst sollten wir die \texttt{tb.vhd} \"andern, um die verschiedenen Funktionen des Befehlsz\"ahlers zu simulieren.
Danach wurde die funktionale Simulation durchgef\"uhrt, in der wir mithilfe der angelegten Signale die Schaltung untersuchten.
Hier sollten wir verstehen, welche Signalfolgen ben\"otigt werden, um gew\"unschte Funktionen des Befehlsz\"ahlers nutzen zu k\"onnen.
Daraufhin wurden die VHDL-Beschreibungen synthetisiert, implementiert und kurz mit dem FPGA-Editor auf die Anzahl der Komponenten untersucht.
Dann wurde dann die zeitbehaftete Simulation durchgef\"uhrt. Es wurde wieder die Schaltung mithilfe der angelegten Signalen analysiert und Zeitmessungen durchgef\"uhrt.
Hier musste die \texttt{tb.vhd} ein weiteres mal bearbeitet werden, da hier die technisch bedingten Zeitunterschiede ber\"ucksichtigt werden mussten.
Schlie\ss lich wurde das FPGA-System konfiguriert und die Funktion des Befehlsz\"ahlers mithilfe des Programmes \texttt{prob} getestet.


\newpage

\section{Erprobung}

\subsection{Funktionale Simulation}
Im folgenden werden erst die verschiedenen Signale aufgelistet und dessen Funktion beschrieben.
\begin{description}
\item [\texttt{const\_sel}] wird genutzt, um eine Konstante aus $\{0x00, 0x01, 0x04\}$ auszuw\"ahlen.
\item [\texttt{/const\_o1\_en}] wird genutzt, um die durch \texttt{const\_sel} ausgew\"ahlte Konstante auf den S1\_Bus zu legen.
\item [\texttt{/const\_o2\_en}] wird genutzt, um eine 0 auf den S2\_Bus zu legen.
\item [\texttt{PC\_latch\_en}] wird genutzt, um das Zulassen von Eingaben aus dem \texttt{Dest\_bus} zu aktivieren oder deaktivieren.
\item [\texttt{PC\_out\_en}] wird genutzt, um den Ausgang des Befehlsz\"ahlers hochohmig zu schalten.
\item [\texttt{PC\_addr\_out}] ist der Bus, um den Inhalt des Befehlsz\"ahlers an den Speicher zu leiten.
\item [\texttt{PHI\_1}] ist ein Takt, der die Eingabelatches f\"ur die ALU steuert, mit der der Befehlsz\"ahler hochz\"ahlt.
\item [\texttt{PHI\_2}] ist ein Takt, der um eine halbe Periode von \texttt{PHI\_1} verschoben ist und den Latch des Befehlsz\"ahlers steuert.
\end{description}

In der Tabelle \ref{funktion} werden die verschiedenen Funktionen des Befehlsz\"ahlers und dessen ben\"otigten Signalfolgen aufgezeigt.
Die Notation der Signalmuster wurde hier aus dem Kapitel 2.2 auf Seite 4 aus der Versuchsvorbereitung entnommen.
Zum Anwenden der gew\"unschten \"Anderung gem\"a\ss \ Tabelle \ref{funktion} muss anschlie\ss end jeweils ein Takt durchgef\"uhrt werden.

\begin{table}[h]
    \centering
    \begin{tabular}{l|r}
    \hline
    Funktion       & Signalmuster \\
    \hline
    auf 0 setzen   & 1100 0001 \\
    um 0 erh\"ohen & 0110 0001 \\
    um 1 erh\"ohen & 0110 0101 \\
    um 4 erh\"ohen & 0110 1001 \\
    \hline
    \end{tabular}
    \caption{Funktionen des Datenpfades}
    \label{funktion}
\end{table}

In diesem Experiment wurde mit einer 8-Bit-Architektur gearbeitet, was bedeutet, dass 8-Bit-Busse benutzt werden.
Dies impliziert, dass auch Befehle jeweils 8 Bit lang sind.
M\"ochte man nun den jeweils n\"achsten Befehl aus dem Speicher laden, so muss man den Befehlsz\"ahler um 1 erh\"ohen, da Speicher f\"ur gew\"ohnlich byteadressierbar sind.
Dies ist also die Standardfunktion des Befehlsz\"ahlers.
Bei einer 32-Bit-Architektur m\"usste man entsprechend 4 zur aktuellen Adresse addieren, um den n\"achsten Befehl zu erhalten.

Da der Zustand des Latches beim Start undefiniert ist, muss dieser in einem ersten Schritt mit einer 0 \"uberschrieben werden.
Tabelle \ref{funktion} beschreibt, wie dies geht.

Wir haben die Testbench anschlie\ss end so abge\"andert, dass die aufgelisteten Funktionen des Befehlsz\"ahlers in der funktionalen Analyse n\"aher analysiert werden k\"onnen.
Eine \"Ubersicht der funktionalen Simulation kann auf der Anlage 1 betrachtet werden.
Durch die funktionale Analyse konnten wir Folgendes feststellen:

\begin{itemize}
  \item Daten gelangen vom Konstantenspeicher auf den \texttt{S1\_Bus}, indem man mit dem Signal \texttt{/const\_o1\_en} den Ausgang aktiviert und per \texttt{const\_sel} die Konstante w\"ahlt. Auf den \texttt{S2\_Bus} kann mit \texttt{/const\_o2\_en} nur eine 0 angelegt werden. Au\ss erdem muss mit \texttt{PC\_out\_en} der Ausgang des Befehlsz\"ahlers hochohmig geschalten werden, falls man vom Konstantenspeicher auf den \texttt{S2\_Bus} \"ubertragen will, um Konflikte zu vermeiden.
  \item Das Resultat der ALU ist stabil, nachdem der Takt \texttt{PHI\_1} ausgelaufen ist. Dann liegen n\"amlich in den Eingangslatches die beiden letzten Werte, die kurz vom dem Umschalten des Taktes auf 0 auf den Latches lagen (Siehe Transparenz von Latches, wie in Versuch A1 untersucht worden ist).
  \item Der Inhalt des Befehlsz\"ahlers ist stabil, wenn entweder \texttt{PC\_latch\_en} oder \texttt{PHI\_2} auf 0 geschaltet wird.
  \item Der Inhalt des Befehlsz\"ahlers wird konstant auf den Adress-Bus gegeben.
  \item der Inhalt des Befehlsz\"ahlers wird auf den \texttt{S2\_Bus} gelegt, wenn \texttt{/PC\_out\_en} auf 0 liegt. Dabei sollte ber\"ucksichtigt werden, dass \texttt{/const\_o2\_en} auf 1 liegt, damit es keine Komplikationen zwischen den Ausgaben des Konstantenspeichers und des Befehlsz\"ahler gibt.
\end{itemize}


\subsection{Synthese und Implementierung}
Diesen Schritt hat wieder \texttt{imp.sh} \"ubernommen.
Tabelle \ref{netzliste} zeigt die Komponenten, die dabei verwendet werden.

\begin{table}[h]
    \centering
    \begin{tabular}{l|l}
    \hline
    Typ     & Anzahl  \\
    \hline
    IBUF    & 8       \\ % Steuerleitungen
    BUF     & 2       \\ % phi1, 2
    IOB     & 16      \\ % pc_adress_out, Steuerleitungen
    SLICEL  & 9       \\
    \hline
    \end{tabular}
    \caption{FPGA-Komponenten der Netzliste}
    \label{netzliste}
\end{table}

\subsection{Zeitsimulation}
Dieselbe Testbench ohne \"Anderung zu \"ubernehmen hat nicht funktioniert. Die resultierende Simulation mit dieser Testbench kann in Anlage 2 betrachtet werden.
Zum einen gibt es nun die in den Arbeitsanweisungen erw\"ahnte Initialisierungszeit.
Zum anderen liegt auf dem \texttt{PC\_addr\_out}-Bus konstant der Wert 0 statt 0, 1, 5, 0 wie bisher.
Die Vermutung liegt nahe, dass die voreingestellte Zeit von 10ns pro Viertelperiode zu kurz ist.
Deshalb haben wir den Wert zun\"achst auf P:= 25ns ge\"andert, den maximalen Wert, den 19 Vierteltakte in 500 ns zulassen. Die Simulation mit dieser Zeitanpassung liegt ausgedruckt als Anlage 3 bei.
Daraufhin korrupiert der Wert von \texttt{PC\_addr\_out} kurz nach Takt 14.
An dieser Stelle hatten wir bisher den Fall, dass \texttt{PC\_out\_en} an dieser Stelle hochohmig geschalten wird, was mit der fallenden Flanke von \texttt{PHI\_1} zusammenf\"allt.
Wir vermuteten den Fehler an dieser Stelle, da die ALU hier m\"oglicherweise mit defekten Werten rechnete.
Nachdem wir \texttt{PC\_out\_en} und dann auch \texttt{const\_o2\_en} jeweils einen Vierteltakt sp\"ater umgeschaltet haben, lief alles wie geplant.
Die ausgedruckte Simulation mit dieser Anpassung liegt als Anlage 4 bei.

In dieser Ansicht sind die Busse S1 und S2 sowie \texttt{Dest\_Bus} nicht mehr einsehbar.
Aus diesem Grund muss man im Folgenden eine genaue Vorstellung von den Daten haben, die auf ihnen liegen.
Man kann nicht mehr direkt \"uberpr\"ufen, ob die eigenen Eingaben das richtige Ergebnis liefern, sondern kann im Fehlfall h\"ochstens ein paar Takte sp\"ater feststellen, \emph{dass} etwas falsch gelaufen ist.

Daraufhin haben wir die Verz\"ogerungen zwischen \texttt{PHI\_2 $\uparrow$} und \texttt{PC\_addr\_out} f\"ur verschiede Werte des Befehlsz\"ahlers gemessen.
Die Messungen sind in Tabelle \ref{vergleich} dargestellt.
Eine Beispielmessung liegt auch ausgedruckt als Anlage 5 vor.
Die Werte sind sich dabei sehr \"ahnlich, aber nicht identisch.
Unsere Erkl\"arung daf\"ur ist, dass es 10.226ns ben\"otigt, um das unterste Bit zu ver\"andern und noch einmal 0.019ns mehr, um das drittniedrigste Bit umzuschalten.

\begin{table}[h]
    \centering
    \begin{tabular}{l|l|c}
    \hline
    Wert Dest\_Bus & Zeit in ns & Bit\"anderung (unterster Teil)\\
    \hline
    0              & 10.245     & 101 -> 000 \\ % 5->0
    1              & 10.226     & 000 -> 001 \\ % 0->1
    5              & 10.245     & 001 -> 101 \\ % 1->5 (normale Latch-Zeiten?)
    \hline
    \end{tabular}
    \caption{Vergleich der Zeiten zwischen funktionaler und zeitbehafteter Simulation in ns.}
    \label{vergleich}
\end{table}

\subsection{Erprobung der Experimentalschaltung}

Hier wurde wieder, analog zum Versuch A1, das Experimentalsystem mit dem Befehl \texttt{konf fpga.bit} konfiguriert, woraufhin wir mit dem Programm \texttt{prog} das System testen konnten.

Wir haben dabei in Tabelle \ref{erprobung} unsere Eingaben des Programmes dokumentiert.
In der Tabelle bedeutet "xy", dass im Befehlsz\"ahler beliebige Werte stehen k\"onnten.
Zeilen 2 - 5 demonstrieren dabei das Setzen des Befehlsz\"ahlers auf 0x00.
In den Zeilen 6 - 9 wird dann das Hochz\"ahlen des Befehlsz\"ahlers konfiguriert.
In Zeile 7 kann dabei die Schrittweite des Hochz\"ahlens eingestellt werden.
Zur Demonstration haben wir 0x01 gew\"ahlt, aber auch andere Schrittweiten ausgef\"uhrt.
Dabei traten die erwarteten Ergebnisse auf.

\newcommand{\off}{\fullmoon}
\newcommand{\on}{\newmoon}

\begin{table}[h]
    \centering
    \caption{Erprobung der Experimentalschaltung mit unserer Versuchsabfolge.
    \on\ bedeutet LED an.}
    \begin{tabular}{l|l|l|l}
    \hline
    Schritt & \texttt{prog}-Kommado & LED-Anzeige  & 7-Segment-Anzeige \\
    \hline
    1  & sple & \on  \on  \on  \on   \off \off \off \on            & xy \\
    2  & cs0  & \on  \on  \on  \on   \off \off \off \on            & xy \\
    3  & rce1 & \on  \on  \on  \off  \off \off \off \on            & xy \\
    4  & rce2 & \on  \on  \off \off  \off \off \off \on            & xy \\
    5  & rck  & \on  \on  \off \off  \off \off \on  \off           & 00 \\
    6  & sce2 & \on  \on  \on  \off  \off \off \on  \off           & 00 \\
    7  & cs1  & \on  \on  \on  \off  \off \on  \on  \off           & 00 \\
    8  & rpoe & \off \on  \on  \off  \off \on  \on  \off           & 00 \\
    9  & sck  & \off \on  \on  \off  \off \on  \off \on            & 00 \\
    10 & rck  & \off \on  \on  \off  \off \on  \on  \off           & 01 \\
    11 & sck  & \off \on  \on  \off  \off \on  \off \on            & 01 \\
    12 & rck  & \off \on  \on  \off  \off \on  \on  \off           & 02 \\

    \hline
    \end{tabular}
    \label{erprobung}
\end{table}

\newpage

\section{Schlussfolgerungen}
In diesem Experiment haben wir erfolgreich ein Teil des Datenpfades, insbesondere den Befehlsz\"ahler untersuchen k\"onnen. Hierbei konnten wir beobachten, wie ein Befehlsz\"ahler funktioniert und welche Signalfolgen f\"ur welche Funktionen ben\"otigt werden. Dabei war die zeitbehaftete Analyse besonders interessant. Einerseits, weil die Testbench neu bearbeitet werden musste, um \"uberhaupt eine korrekte Analyse durchf\"uhren zu k\"onnen. Andererseits, weil wir die Busse nicht konkret beobachten konnten, welches die Analyse etwas erschwert hat.


\end{document}
