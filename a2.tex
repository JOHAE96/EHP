
\documentclass[12pt,a4paper]{scrartcl}

\usepackage{fancyhdr}
\usepackage{times}
\usepackage[T1]{fontenc}
\usepackage[latin1]{inputenc}
\usepackage[ngerman]{babel}
\usepackage[right=3cm]{geometry}
\usepackage{wasysym} %for moon symbols
\parindent=0pt
\parskip=1em plus 2pt minus 1pt

\topskip-2cm
\textheight25cm
\pagestyle{fancy}
\fancyhead{}
\fancyfoot{}
\rhead{\thepage}

\begin{document}
\thispagestyle{empty}

\begin{center}
  \LARGE
  Praktische \"Ubungen - \\
  Experimentelle Hardwareprojekte \\
  \bigskip
  \Large
  Versuchsprotokoll
\end{center}

\vspace{1em}
Versuch: A2 - Befehlsze\"ahler eines RISC-Prozessors

Versuchsdatum und -zeit: 16. Mai 2018, 14 - 17 Uhr

Betreuer: Volker D\"orsing

\vspace{1em}
Name, Studiengang, Mat.-Nr.: Alexander K\"uhnle, B.Sc. Informatik, 165692

Email: alexander.kuehnle@uni-jena.de

\vspace{1em}
Name, Studiengang, Mat.-Nr.: Mark Umnus, B.Sc. Informatik, 167419

Email: mark.umnus@uni-jena.de

\vspace*{1cm}
%% \mbox{}\\
\hrule
\vspace*{1cm}
{\Large  Eigenst\"andigkeitserkl\"arung }

Hiermit versichern wir, dass wir das Protokoll selbstst\"andig verfasst
und keine anderen Quellen und Hilfsmittel als die angegebenen benutzt
haben. Im Falle einer Zuwiderhandlung erkennen wir an, dass unser Protokoll
als nicht bestanden bewertet wird und damit das Modul ``Experimentelle
Hardwareprojekte'' als nicht bestanden bewertet wird. \\
Dar\"uberhinaus ist uns klar, dass jede Zuwiderhandlung ausnahmslos dem
Rechtsamt der FSU gemeldet wird, woraus weitere Konsequenzen resultieren
k\"onnen. \\

Unterschriften: \\
\hspace*{4cm} ........................................
\hspace{2cm} ........................................  \\

\hrule

\vspace*{0.3cm}
\textbf{Vom Betreuer auszuf\"ullen:}

Vorbereitung/Kolloquium:

Durchf\"uhrung:

Protokoll:

Gesamtbewertung:
\clearpage


% Hier geht das Protokoll los...

\section{Vorbereitung}
In Versuch A2 wird der Befehlsz\"ahler der DLXJ-Prozessorarchitektur n\"aher betrachtet.
Ziel des Versuches ist es, die ben\"otigten Signalfolgen zu verstehen, damit der Befehlsz\"ahler korrekt funktioniert.
Dazu sollten wir in der funktionalen und zeitbehafteten Simulation den Befehlsz\"ahler untersuchen und verstehen, unter welchen Vorraussetzungen
die Funktionen des Befehlsz\"ahlers ausgef\"uhrt werden k\"onnen.

\section{Vorgehensweise}

Uns lagen f\"ur den Versuch mehrere Dateien vor: F\"ur die Testumgebung eine \texttt{tb.vhd} und weitere \texttt{*.vhd} f\"ur die Beschreibung der Komponenten des Befehlsz\"ahlers. Au\ss erdem wurden uns, analog zu Versuch A1, die Skripte \texttt{fsim.sh} und \texttt{tsim.sh} f\"ur die Analysen und \texttt{imp.sh} für Synthese und Implementierung, zur Verf\"ugung gestellt.

Zuerst sollten wir die \texttt{tb.vhd} \"andern, um die verschiedenen Funktionen des Befehlsz\"ahlers simulieren. Danach wurde die funktionale Simulation durchgef\"uhrt, in der wir mithilfe der angelegten Signale die Schaltung untersuchten. Hier sollten wir verstehen, welche Signalfolgen ben\"otigt werden, um gew\"unschte Funktionen des Befehlsz\"ahlers nutzen zu k\"onnen. Daraufhin wurden die VHDL-Beschreibungen implementiert, synthetisiert und kurz mit dem FPGA-Editor auf die Anzahl der Komponenten untersucht. Dann wurde dann die zeitbehaftete Simulation durchgef\"uhrt. Es wurden wieder die Schaltung mithilfe der angelegten Signalen analysiert und Zeitmessungen durchgef\"uhrt. Hier musste die \texttt{tb.vhd} ein weiteres mal bearbeitet werden, da hier die technisch bedingten Zeitunterschiede ber\"ucksichtigt werden mussten. Schlie\ss lich wurde das FPGA-System konfiguriert und die Funktion des Befehlsz\"ahlers mithilfe des Programmes \texttt{prob} getestet.

\newpage

\section{Erprobung}

\subsection{Funktionale Simulation}
\begin{description}
\item [const\_sel] wird genutzt, um eine Konstante aus $\{0x00, 0x01, 0x04\}$ auszuw\"ahlen.
\item [/const\_o1\_en] wird genutzt, um die durch const\_sel ausgew\"ahlte Konstante auf den S1\_Bus zu legen.
\end{description}


\begin{table}[h]
    \centering
    \begin{tabular}{l|r}
    \hline
    Funktion       & Signalmuster \\
    \hline
    auf 0 setzen   & 1100 0001 \\
    um 0 erh\"ohen & 0110 0001 \\
    um 1 erh\"ohen & 0110 0101 \\
    um 4 erh\"ohen & 0110 1001 \\
    \hline
    \end{tabular}
    \caption{Funktionen des Datenpfades}
    \label{funktion}
\end{table}


Da der Zustand des Latches beim Start undefiniert ist, muss dieser in einem ersten Schritt mit einer 0 \"uberschrieben werden.
Tabelle \ref{funktion} beschreibt, wie dies geht.
Wir haben die Testbench anschlie\ss end so ange\"andert, dass

\subsection{Synthese und Implementierung}
Diesen Schritt hat wieder \texttt{imp.sh} \"ubernommen.

\begin{table}[h]
    \centering
    \begin{tabular}{l|l}
    \hline
    Typ     & Anzahl  \\
    \hline
    IBUF    & 8       \\ % Steuerleitungen
    BUF     & 2       \\ % phi1, 2
    IOB     & 16      \\ % pc_adress_out, Steuerleitungen
    SLICEL  & 9       \\
    \hline
    \end{tabular}
    \caption{FPGA-Komponenten der Netzliste}
    \label{netzliste}
\end{table}

\subsection{Zeitsimulation}
Dieselbe Testbench ohne \"Anderung zu \"ubernehmen hat nicht funktioniert.
Zum einen gibt es nun die in den Arbeitsanweisungen erw\"ahnte Initialisierungszeit.
Zum anderen liegt auf dem \texttt{adress\_out}-Bus konstant der Wert 0 statt 0, 1, 5, 0 wie bisher.
Die Vermutung liegt nahe, dass die voreingestellte Zeit von 10ns pro Viertelperiode zu kurz ist.
Deshalb haben wir den Wert zun\"achst auf P:= 25ns ge\"andert, den maximalen Wert, den 19 Vierteltakte in 500 ns zulassen.
Daraufhin korrupiert der Wert von address\_out kurz nach Takt 14.
An dieser Stelle hatten wir bisher den Fall, dass PC\_out\_en an dieser Stelle hochohmig geschalten wird, was mit der fallenden Flanke von Phi1 zusammenf\"allt.
Wir vermuteten den Fehler an dieser Stelle, da die ALU hier m\"oglicherweise mit defekten Werten rechnete.
Nachdem wir PC\_out\_en und dann auch const\_o2\_en jeweils einen Vierteltakt sp\"ater umgeschaltet haben, lief alles wie geplant.

In dieser Ansicht sind die Busse S1 und S2 sowie Dest\_Bus nicht mehr einsehbar.
Aus diesem Grund muss man im Folgenden eine genaue Vorstellung von den Daten haben, die auf ihnen liegen.
Man kann nicht mehr direkt \"uberpr\"ufen, ob die eigenen Eingaben das richtige Ergebnis liefern, sondern kann im Fehlfall h\"ochstens ein paar Takte sp\"ater feststellen, \emph{dass} etwas falsch gelaufen ist.

\begin{table}[h]
    \centering
    \begin{tabular}{l|l}
    \hline
    Wert Dest\_Bus & Zeit in ns \\
    \hline
    0              & 10.245     \\ % 5->0
    1              & 10.226     \\ % 0->1
    5              & 10.245     \\ % 1->5 (normale Latch-Zeiten?)
    \hline
    \end{tabular}
    \caption{Vergleich der Zeiten zwischen funktionaler und zeitbehafteter Simulation in ns}
    \label{vergleich}
\end{table}

\subsection{Erprobung der Experimentalschaltung}

\newcommand{\off}{\fullmoon}
\newcommand{\on}{\newmoon}

\begin{table}[h]
    \centering
    \caption{Erprobung der Experimentalschaltung mit unserer Versuchsabfolge.
    \on\ bedeutet LED an.}
    \begin{tabular}{l|l|l}
    \hline
    Stelle & \texttt{tst}-Kommado & LED-Anzeige    \\
    \hline
    1      & sce               & \off \off \off \off  \\
    2      & wd 0x6            & \off \off \off \off  \\
    3      & sck               & \off \on  \on  \off \\
    4      & wd 0xc            & \on  \on  \off \off \\
    5      & soe               & \off \off \off \off \\
    6      & rck               & \off \off \off \off \\
    7      & roe               & \on  \on  \off \off  \\
    8      & sck               & \on  \on  \off \off \\
    9      & rce               & \on  \on  \off \off \\
    10     & wd 0xf            & \on  \on  \off \off \\
    11     & sce               & \on  \on  \on  \on  \\
    \hline
    \end{tabular}
    \label{erprobung}
\end{table}



\section{Schlussfolgerungen}


\end{document}
